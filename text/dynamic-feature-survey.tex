\matlab supports many heavily dynamic features that are problematic for static
analysis, and which at present are either ignored or rejected by different
components of the McLab toolkit. These include

\begin{itemize}

\item scripts, which have error prone scoping semantics compared to functions

\item arbitrary dynamic code evaluation via \code{eval}

\item dynamic function calls via \code{feval}

\item dynamic workspace manipulation either by deleting variables via
\code{clear} or \code{clearvars} or assigning to them via \code{assignin} or
\code{evalin}

\item dynamic workspace inspection via \code{exist}, \code{who} or \code{whos}

\item dynamic modification of the function lookup path via \code{cd},
\code{path}, \code{addpath}, \code{rmpath}

\end{itemize}

In this chapter, we examine each of these, giving a brief primer on their
semantics and investigating how frequently they occur, and common patterns in
their usage. Similar invstigations for other dynamic programming languages such
as JavaScript \cite{TheEvalThatMenDo} and Ruby \cite{ProfileGuidedStaticTyping}
have shown that in practice, such dynamic features tend to be used in very
restricted ways which aren't actually difficult to reason about statically,
and we find that the same holds for \matlab.

\section{\mcbench}

% TODO mcbench appendix?

The following sections present usage metrics for various dynamic features
supported by \matlab. These were gathered using \mcbench, a tool that that
allows users to perform structural queries against a large body of \matlab
code\footnote{\mcbench is accessible at \url{http://mcbench.cs.mcgill.ca}.}. It
works by storing on the filesystem an XML version of each \matlab file in its
corpus -- the \mclab toolkit includes a facility to serialize \matlab abstract
syntax trees as XML (an example is given in listings \lstref{Lst:McBenchCode}
and \lstref{Lst:McBenchXml}) -- and allowing XPath queries against them. XPath
turns out to be quite a useful sort of domain specific language in this
context, capable of expressing many useful queries. \mcbench also defines a few
XPath extension functions and predicates in order to make certain queries more
natural. An initial (obsolete) implementation of \mcbench is described in more
detail in \cite{SoroushThesis}.

\lstinputlisting[
  language=Matlab,
  caption={A simple \matlab function...},
  label=Lst:McBenchCode,
]{code/mcbench.m}

\lstinputlisting[
  language=XML,
  caption={...and its XML serialization.},
  label=Lst:McBenchXml,
  float
]{code/mcbench.xml}

\mcbench's corpus of \matlab code consists of projects downloaded from the
\matlab Central File Exchange, an online repository run by MathWorks where
\matlab programmers share their code. Of these, we selected the 5000 top rated
projects and the 5000 most downloaded projects. After discarding duplicates and
programs that couldn't be parsed by our \matlab frontend, we were left with
4099 projects, containing together 24565 functions and 2955 scripts. The
projects cover a wide variety of application areas, and include both library
and application code. The projects vary in size from single files to several
hundred; a rough size distribution is given in \tableref{tab:1}.

\begin{table}
\centering
\begin{tabular}{| l | l |}
\hline
Project size in files & Number of projects \\ \hline
Single (1) & 2444 \\ \hline
Small (2-9) & 1371 \\ \hline
Medium (10-49) & 261 \\ \hline
Large (50-99) & 17 \\ \hline
Very large (100+) & 6 \\ \hline
\end{tabular}
\caption{File count per project distribution.}
\label{tab:1}
\end{table}

\section{Scripts}

\matlab scripts are files containing a sequence of statements (as opposed to
function definitions). Like functions, scripts can be invoked, executing each
of their statements in turn; this is done using the same syntax as for function
calls. Unlike functions, however, scripts don't execute in their own workspace
(or scope); instead, scripts execute directly in their caller's workspace, and
thus have read and write access to any variables in scope there. Also, scripts
don't have an explicit parameter-passing mechanism; rather, one can simply
ensure all the required variables exist in the calling workspace.

Within a function, a use of an identifier that isn't declared in the same file
(either as an input, output, global, persistent, or local variable, or a
subfunction or nested function) must refer to a named function somewhere on
\matlab's search path. Inside a script, however, an undeclared identifier could
also refer to a variable in the workspace of the script's caller. This
complicates intraprocedural static analysis for scripts.

As an aside, one surprising consequence of the way scripts are executed is that
scripts almost behave as though they were inlined in the calling code. For
example, a script could contain a break or continue statement, and if that
script were invoked inside a loop, then those statements would apply to that
loop.

% TODO(isbadawi): Numbers are off for functions that take no arguments.
% We run queries like //ParameterizedExpr[is_call('...')] but need also
% NameExpr...

\section{\texttt{eval} and variants} \label{sec:Eval}

The \code{eval} function evaluates \matlab code passed to it as a string. This
code is almost arbitrary, and can have side-effects, such as the creation of
new variables, although function definitions are not allowed. Calls to
\code{eval} that occur within anonymous functions, nested functions, or
functions containing nested functions are not allowed to create new variables.
Any outputs from the evaluated code are returned via \matlab's variable-length
output argument list mechanism.

\code{eval} also permits a two-argument form. Normally, if the evaluation
throws an error, it is silently ignored; if a second argument is passed, then
it is evaluated instead in case of errors. This two-argument form is deprecated
and undocumented and can almost always be replaced by \matlab's \code{try ...
catch} exception handling mechanism. Nevertheless, it still enjoys some use.

We start by finding all calls to \code{eval} using the query from
\lstref{Lst:QueryEval}. (Here \code{is_call} is a \mcbench-specific XPath
extension that leverages kind analysis results to distinguish function calls
from array accesses). This yields 1049 occurrences across 202 projects, or
4.93\% of the projects in the corpus. We inspect these to try and distinguish
different use cases.

\begin{lstlisting}[
  language=Matlab,
  caption={\mcbench query that finds calls to \code{eval}},
  label=Lst:QueryEval,
  float
]
//ParameterizedExpr[is_call('eval')]
\end{lstlisting}


\subsection{Manipulating related variables} \label{sec:ArrayLikeEval}

The most common pattern is to use \code{eval} to create or manipulate sets of
related variables by calling it inside a loop, using the loop variable to
construct its input argument. An example is shown in \lstref{Lst:EvalNoArray};
here, the effect of the loop is to store values in three arrays, \code{n},
\code{s2}, and \code{Z}, which are then used later in the code. Rather than
assigning to elements of these arrays, intermediate variables are created via
calls to \code{eval}, and the arrays are built up incrementally using array
concatenation in the last statement of the loop. The many intermediate
variables created along the way are not referenced again after the loop.

We will call such calls \emph{array-like}; array-like calls to \code{eval} can
almost always be refactored to use arrays, cell arrays or structures.

\lstinputlisting[
  language=Matlab,
  caption={An iterative numerical procedure implemented using repeated calls
      to \code{eval} instead of arrays.},
  label=Lst:EvalNoArray,
  float
]{code/loopeval.m}

Without some sort of string analysis, it is difficult to nail down exactly how
many of the calls follow this pattern. We estimate the amount by searching for
calls to \code{eval} that appear inside a for-loop, and that have as
descendents either calls to \code{num2str} or \code{int2str} where the argument
is the loop variable, or calls to \code{sprintf} where the loop variable
appears somewhere. This is difficult to express in XPath, so we cheat a little
by introducing an extension function called \code{loopvars()}, which computes
the set of loop variable names in the current context by walking up the AST,
finding for statements and inspecting their header. The resulting query is
given in \lstref{Lst:QueryEvalLoopIndex}, reports 300 occurrences across 33
benchmarks, or nearly a third of all calls to \code{eval}.

This is an underestimate, since the query only matches calls where the argument
string is constructed syntactically inside the argument list; for instance, the
query won't match calls where the string interpolation is hidden behind a
function call, or if the call is of the form \code{eval(s)} where \code{s} was
previously defined according to our criteria. It may be possible to annotate
the XML form of the AST with enough semantic information (for instance, UD
chains) to enable more sophisticated queries.

% TODO(isbadawi): do this? Use ids in the XML to add e.g. uses="12,15,53" on
% defs, defs="7,9" on name expressions.

\begin{lstlisting}[
  language=Matlab,
  caption={\mcbench query that estimates calls to eval that use the loop index},
  label=Lst:QueryEvalLoopIndex,
  float
]
//ParameterizedExpr[is_call('eval') and
  count(.//ParameterizedExpr[
    (is_call('num2str', 'int2str') and arg(1)/Name/@nameId=loopvars()) or
    (is_call('sprintf') and .//Name/@nameId=loopvars())
  ]) > 0]
\end{lstlisting}

\subsection{Restricted calls to \code{eval}}

As mentioned, calls to \code{eval} ocurring within anonymous functions, nested
functions, or functions containing nested functions are not allowed to create
variables. Such calls turn out to be relatively uncommon, occurring 31 times
across 18 benchmarks. The relevant query is given in
\lstref{Lst:QueryRestrictedEval}.

\begin{lstlisting}[
  language=Matlab,
  caption={\mcbench query that matches calls to eval occuring within anonymous
      functions, nested functions, or functions containing nested functions},
  label=Lst:QueryRestrictedEval,
  float]
//ParameterizedExpr[is_call('eval') and (
    ancestor::LambdaExpr or
    count(ancestor::Function) > 1 or
    (count(ancestor::Function) = 1 and ancestor::Function//Function)
)]
\end{lstlisting}

\subsection{Two-argument form}

We find only 19 occurrences of \code{eval}'s two-argument form, across 7
benchmarks. The second argument is always a string literal, and contains
typical error-case code like breaking out of a loop, or setting a flag variable
indicating an error occurred, or printing an error message.

In one case, this is used to distinguish between \matlab versions; the first
argument is a string literal containing a call to a builtin function, but with
a signature valid only for \matlab 6.0 or later.

\subsection{\texttt{evalc}}

The \code{evalc} function behaves similarly to \code{eval}, but also captures
any command window output from the evaluation and returns it as a character
array as an additional output argument.

\mcbench only found twelve calls to \code{evalc}. Of these, five are passed
string literals. Four ignore the return value, the only feature distinguishing
\code{evalc} from \code{eval}. In one case the output is captured and
immediately passed to \code{fprintf}. The one legitimate use involves capturing
and parsing the output of the \matlab built-in \code{dbtype} function, which
only displays its output instead of returning it.

\subsection{\texttt{feval}}

The \code{feval} function takes a function --- either a function handle or a
name as a string --- and a variable number of arguments, and evaluates that
function on those arguments, returning whatever the function returns.

We find 457 calls to \code{feval} across 136 benchmarks. At a glance, many of
these benchmarks are concerned with solvers or optimization problems, which are
a natural fit for \code{feval}, allowing callers to pass in arbitrary
functions.

If the first argument is a string literal, or a function handle expression,
then the call to \code{feval} is completely superfluous. We find 12 such calls,
across 5 benchmarks. The relevant query is given in
\lstref{Lst:QueryFevalString}.

\begin{lstlisting}[
  language=Matlab,
  caption={\mcbench query that matches superfluous calls to feval},
  label=Lst:QueryFevalString,
  float
]
//ParameterizedExpr[is_call('feval') and (
    name(arg(1))='StringLiteralExpr' or
    name(arg(1))='FunctionHandleExpr'
)]
\end{lstlisting}

\section{Workspace manipulation}

In \matlab, workspaces (i.e. scopes) store the values of variables. There is a
base workspace on which REPL commands operate, along with a workspace for each
function call (analogous to a stack frame). \matlab supports dynamically
manipulating these workspaces via the builtin functions \code{clear},
\code{clearvars}, \code{evalin}, and \code{assignin}.

% TODO(isbadawi): Need more takeaways from here on, not just numbers

\subsection{\texttt{evalin} and \texttt{assignin}}

The \code{evalin} function behaves similarly to \code{eval}, but takes an extra
parameter indicating the workspace in which the evaluation should happen. This
parameter can be \code{'base'}, indicating the \matlab base workspace, or
\code{'caller'}, indicating the workspace of the caller function.
\code{assignin} is a restricted version of \code{evalin}; rather than support
arbitrary code execution, the function takes a variable name as a string and a
value, together with a workspace to operate on --- again either \code{'base'}
or \code{'caller'} --- and assigns the given value to the given variable in the
given workspace.

We find 169 calls to \code{evalin} across 58 benchmarks. Of these, 124 operate
on the base workspace, and 43 operate on the caller workspace. (Of the two
remaining calls, one takes the workspace to operate on as a function parameter,
and the other seems to contain an error --- the first argument is a reference
to a variable which isn't defined anywhere). For \code{assignin}, we find 176
calls across 52 benchmarks, 167 operating on the base workspace, and only 9 on
the caller workspace. We notice among these a few patterns, but no one majority
case.

\begin{itemize}

  \item Operating on the base workspace is sometimes used in lieu of global
    variables, or in lieu of explicit function parameters.

  \item Another pattern is that of the "setup function"; a function that
    creates a few variables in the base or caller workspace, making them
    available to be used at the \matlab command window.

  \item It seems common for libraries providing programming utilities to
    operate on the caller's workspace. For instance, a common pattern in
    \matlab functions is to use the \code{nargin} or \code{exist} builtin
    functions to tell whether a particular input parameter was passed, and to
    fall back to a default value if not. One library function encapsulates this
    logic, using \code{evalin} to carry it out in the caller's workspace given
    a parameter name and default value. Another library provides a function
    called \code{keep} as a counterpart to the \code{clear} statement; where
    \code{clear} removes the given variables from the workspace, \code{keep}
    removes all but the given variables from the workspace. \code{keep} is
    implemented by scanning the caller's workspace via \code{evalin}, then
    constructing an invocation of the \code{clear} statement to be evaluated in
    the caller's workspace.

\end{itemize}

\subsection{\texttt{clear} and \texttt{clearvars}}

The \code{clear} function is used to remove items from the current workspace,
and providing they're not declared global, freeing them from memory. When
called with no arguments, it removes all variables from the workspace.
Alternatively, it can be passed a variable number of variable names or regular
expressions matching variables names to remove. Finally, it also recognizes
some special parameters that refer to types of names of clear, such as
\code{'all'}, \code{'classes'}, or \code{'global'}. \code{clearvars} is similar
but has even more options to control which names to remove.

We find 1236 calls to \code{clear} across 386 benchmarks, and only 13 calls to
\code{clearvars}. Of the calls to \code{clear}, only one uses the no argument
form, and only one uses the regular expression facility. 103 calls use the
\code{clear all} form. The rest explicitly pass in a sequence of variable names
to remove.

% TODO(isbadawi): LDVARs pushed to ID, ID might be a function

These results seem slightly implausible, since a very common pattern is for
\matlab scripts to begin by clearing the workspace. This discrepancy is caused
by imprecise kind analysis results for scripts \cite{KindAnalysis}. Since
scripts execute in the context of the calling workspace, the conservative
approach is to assume that any given identifier potentially refers to a
variable from the outer scope, rather than a builtin or library function.
Because of this, calls to \code{clear} inside scripts aren't identified as
such, and are assumed to be possible variable references. It may be possible to
devise a more sophisticated interprocedural kind analysis that considers the
possible workspaces that a script may execute in, but this would be difficult
(because for instance, computing a call graph for \matlab is difficult).

Since the difference is significant here, we also consider possible references
to variables called \code{clear} occurring inside scripts. The relevant query
is given in \lstref{Lst:Clear}. This yields 950 occurrences across
346 benchmarks. Of these, 26 use the no-argument form, and 682 use the
\code{clear all} form.

\begin{lstlisting}[
  language=Matlab,
  caption={\mcbench query that matches possible calls to clear},
  label=Lst:Clear,
  float
]
//ParameterizedExpr[
    name(target())='NameExpr' and
    target()/Name/@nameId='clear' and
    target()/@kind='LDVAR' and
    ancestor::Script
]
\end{lstlisting}

\section{Introspection}

\matlab supports dynamic introspection via the \code{exist}, \code{who} and
\code{whos} functions.

\subsection{\texttt{exist}}

The former takes an identifier as a string and checks whether it exists, and if
so what its kind is -- variable, path, MEX-file, Simulink model, builtin,
protected function file, folder or class. A name might exist with more than one
kind, in which case \matlab's documentation specifies a largely arbitrary order
of evaluation that determines which kind is returned. The function also permits
a two-argument form, where the second argument is a string specifying which
kind to check -- either \code{'builtin'}, \code{'class'}, \code{'dir'},
\code{'file'}, or \code{'var'}.

We find 1177 calls to \code{exist} across 377 benchmarks. Of these, 939 are
passed a string literal as the first argument, 241 use the one-argument form,
and 936 use the two-argument form. With the two-argument form, the second
argument is always a string literal; \code{'var'} in 657 cases, \code{'file'}
in 220 cases, \code{'dir'} in 51 cases, and \code{'class'} and \code{'builtin'}
in 4 cases each.

A common pattern is to check for the existence of certain input arguments.
\matlab allows a function to be called with fewer arguments than are specified
in the function's parameter list; in this way, some parameters can be made
optional. We can check for this kind of use by finding calls to \code{'exist'}
where the first parameter is a string literal whose value is equal to the name
of one in the input parameters. The relevant query is given in listing
\lstref{Lst:ExistInputParam}. We find that 481 calls, across 167 benchmarks,
are of this form, which shouldn't be too hard for an interprocedural analysis
(assuming we can resolve the call graph difficulty to begin with) to reason
about statically. It would be possible to rewrite calls of this form to simple
comparisons against \code{nargin}, a builtin \matlab function that returns the
number of input parameters passed in to the caller's enclosing function, but
it's not clear that this would be worthwhile, or easier to deal with in any
significant way.

\begin{lstlisting}[
  language=Matlab,
  caption={\mcbench query that matches calls to exist that check for the
      existence of input parameters},
  label=Lst:ExistInputParam,
  float
]
//ParameterizedExpr[is_call('exist') and
  arg(1)/@value = ancestor::Function/InputParamList/Name/@nameId
]
\end{lstlisting}

\subsection{\texttt{who} and \texttt{whos}}

The \code{who} function can be used to list variables, either in the current
workspace, a given m-file, or the global scope. The \code{whos} variant behaves
similarly but also includes information about the sizes and types of each
variable. These functions are useful during interactive development at the
command prompt, but, perhaps unsurprisingly, they turn out not to be as popular
inside functions and scripts, with only 4 calls to the former and 23 to the
latter.

\section{Lookup path modification}

When the \matlab runtime needs to look up a function or script, it searches
first the current directory the process is executing in, then the \matlab search
path, a set of directories containing \matlab code. Both of these can be
modified dynamically at runtime; the \code{cd} function changes the current
directory, and the \code{path}, \code{addpath} and \code{rmpath} functions can
be used to modify the search path. Use of these functions can complicate call
graph construction; the state of the filesystem must be taken into account
somehow.

The \code{cd} function changes the current directory. It takes a single string
argument representing the new directory; this can be an absolute or relative
path, possibly containing symbolic \code{..} or \code{.} notation, denoting a
parent directory or a current directory, respectively. \code{cd} can also be
invoked without arguments. If the call site includes an output argument, then
\code{cd} returns the absolute path to the present working directory as a
string. If not, it writes the path to standard output.

We find 187 calls to \code{cd} across 64 benchmarks. Most often, \code{cd} is
used to get at data that the program depends on; the working directory is
changed to a data folder, and functions such as \code{load} or \code{textread}
are then used to read in the data, passing in only the file name, without any
leading directory names. In many cases, \code{cd} is called again immediately
after loading the data, so as to restore the original working directory. Rather
than carrying out this sort of dance, it would be preferable to load the
required data using a path relative to the starting working directory.

The \code{addpath} function can be used to insert directories either at the
start or end of the \matlab search path. It takes a variable number of strings
as arguments, optionally followed by the string \code{'-begin'} or
\code{'-end'}, denoting whether to insert the directories at the start or end
of the path. The default behavior is to add them at the start. It returns the
old path.

The \code{rmpath} function removes a directory from the search path. It takes a
single string argument and returns nothing.

The \code{path} function is more powerful; it can be used to add paths to the
start or end of the search path, but it can also be passed a string array
representing a list of folders to replace the search path entirely. It returns
the old path. This function can also be used to read the path, without
modifying it.

Finally, there also exists a \code{restoredefaultpath} function that discards
any modifications to the search path, and a \code{savepath} function that
persists any modifications to the search path to be used by future \matlab
sessions.

We find 20 calls to \code{addpath} across 16 benchmarks, 4 calls to
\code{rmpath} across 4 benchmarks, 31 calls to \code{path} across 15 benchmarks
-- 18 of which use the two-argument form equivalent to \code{addpath}, and 13
the one-argument form replacing the path. We find no calls to
\code{restoredefaultpath} or \code{savepath}.

\section{Motivation for eliminating uses of dynamic features}

While dynamic features are very powerful, for instance enabling metaprogramming
techniques that aren't otherwise possible, they also have many drawbacks.

\subsection{Impact on static analysis and program comprehension}

Compilers and related tools rely on various static analyses to estimate the
runtime behavior of programs. Such analyses are typically conservative,
preferring to deduce weaker properties that are guaranteed to always hold for
all inputs over stronger properties that might not always hold, even if the
latter would be more useful. This ensures the soundness of any ensuing program
transformations (such as optimizations) predicated on those analyses.

Highly dynamic features tend to confuse traditional static analysis techniques.
For example, a call to \code{eval} can have nearly arbitrary side effects.
Without knowing the exact values passed as arguments -- information which is
rarely knowable statically -- encountering a call to \code{eval} forces a
conservative analysis to invalidate all the knowledge it has gathered thus far
and fall back on safe assumptions. As clients consume the analysis information,
this imprecision leads to missed optimization opportunities in compilers,
crippled code navigation features in IDEs, and spurious warnings in static
analyzers, among others.

\code{eval} is an easy example to pick on, being the most general and powerful
dynamic construct, but other dynamic features also lead to imprecision.
Changing the function lookup path obsoletes static information gathered about
the call graph. Reaching into another function's workspace and assigning to a
variable invalidates information about identifier lookup, making it hard to
distinguish between array accesses and function calls. Loading data via calls
to the \code{load} function complicates reasoning about the shapes of arrays.

\subsection{Performance}

One of the main observations of this section is that many uses of dynamic
features are completely superfluous, and could be equivalently written using
simpler, less powerful constructs of the language. In those cases, it makes
sense to consider the performance characteristics of the two approaches.

As an example, one of the principal patterns identified concerns using repeated
calls to the \code{eval} family of functions in a loop in lieu of using arrays.
The code in \lstref{Lst:EvalNoArray} calls \code{eval}, invoking the full
machinery of the interpreter, no fewer than eight times per loop iteration,
even though each of these could be replaced by two or three array operations.
This has a huge impact on performance. As a preliminary test, we manually
replaced each call to \code{eval} in that benchmark with equivalent code using
arrays. The transformed benchmark achieved a 56x speedup over the original.

Beyond interpreter overhead, code that restricts itself to features that are
easier to reason about statically is apt to be better understood by compilers,
which rely on sophisticated static analysis to identify candidates for
optimization. This is not irrelevant; several \matlab compilers exist, and
recent versions of the proprietary \matlab implementation include a JIT
compiler. % TODO(isbadawi): Could cite things here

\section{Related work}

\subsection{Dynamic feature survey}

This study is inspired in part by Furr et al.'s work on profile-guided static
typing for Ruby \cite{ProfileGuidedStaticTyping}, where one of the major
insights was that although highly dynamic semantics can theoretically lead to
code which is hard to reason about, dynamic features tend to be used is much
more restricted ways in practice. Similar studies done for JavaScript by
Ratanaworabhan et al. \cite{JSMeter} and Richards et al.
\cite{TheEvalThatMenDo,DynamicBehaviorJavaScript} have reached similar
conclusions. Because JavaScript is so ubiquitous on the web, one can instrument
a web browser and visit popular web pages to obtain all sorts of dynamic
metrics. The same is unfortunately not true in our case; although we have easy
access to a large body of \matlab code, it is not trivial in general to
determine how to execute it, which is why we had to restrict ourselves to
static metrics.

\subsection{Dynamic feature elimination}

Furr et al. \cite{ProfileGuidedStaticTyping} present a dynamic profiling
approach to eliminate uses of Ruby's dynamic features like \code{eval},
\code{send} and \code{method\_missing}, with the aim is making code more
amenable to static analysis. They use a profiling run to gather information at
each dynamic use site, for instance the actual arguments passed to \code{eval},
or the actual missing methods for which \code{method\_missing} was invoked, and
use this to replace the usage sites with specialized code for each observed
argument (along with runtime checks to emit warnings in case the arguments used
don't conform to the profiling run).

In the wake of the abovementioned studies conducted by Richards et al.
\cite{DynamicBehaviorJavaScript, TheEvalThatMenDo}, there has been some work on
eliminating uses of \code{eval} in JavaScript. The common thread is exploiting
patterns in the way \code{eval} tends to be used, be it for parsing JSON
strings, executing third party code, or accessing or modifying object
properties with computed names.

Jensen et al.'s Unevalizer tool \cite{RemedyingTheEvalThatMenDo} integrates a
refactoring component with a static whole-program value dataflow analysis. When
the analysis detects dataflow into a possible call to \code{eval}, it triggers
the refactoring component, passing in all the information it has gathered so
far about the values of variables. Guided by knowledge of common patterns, the
refactoring component attempts to replace the code with a static equivalent if
possible. If so, the analysis resumes until all calls to \code{eval} are
eliminated; if not, the tool aborts with an error. Their approach enjoys
moderate success, but is hindered by the imprecision of their static analysis
in various cases.

Meawad et al.'s \cite{EvalBegone} Evalorizer tool instruments all calls to
\code{eval}, intercepting and logging all the strings passed at each call site.
For each particular call site, each argument string is taken to be valid
JavaScript and parsed into an AST. The different ASTs are then merged together
and generalized in different ways in order to construct a "recognizer" tree
associated with that call site, which is then used to generate patches
containing replacement code. The replacement code contains a runtime guard
clause checking that the passed in argument matches the pattern observed for
this call site -- this can be done with a regular expression in most cases. If
it does, then safer but equivalent code is run; if not, a fallback case
contains the original call to \code{eval} (or optionally an exception, if the
tool is so configured). This approach turns out to be fruitful, and the patched
code runs with minimal performance overhead.
