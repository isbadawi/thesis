{\sc Matlab}\textsuperscript{\textregistered} est un langage de programmation
dynamique populaire chez les scientifiques. L'utilisateur typique de \matlab
n'est pas un programmeur professionel; \matlab est
principalement utilis\'{e} par des scientifiques, des ing\'{e}nieurs et des
\'{e}tudiants, et doit sa popularit\'{e} en grande mesure \`{a} sa syntaxe de
haut niveau et \`{a} son \'{e}ventail de librairies pour toutes sortes de
domaines dans les sciences. Le manque d'\'{e}xperience des programmeurs
\matlab, combin\'{e} \`{a} ses s\'{e}mantiques mal sp\'{e}cifi\'{e}es et
souvent contraires \`{a} l'intuition, m\`{e}ne \`{a} du code \matlab qui est
difficile \`{a} comprendre et maintenir.

Dans cette th\`{e}se, nous pr\'{e}sentons McIDE, un environnement de
d\'{e}veloppement int\'{e}gr\'{e} pour \matlab. McIDE fournit des outils visant
\`{a} aider les programmeurs \matlab \`{a} \'{e}crire de meilleurs programmes,
incluant des remaniements automatiques et des fonctions de navigation de code,
par exemple permettant de sauter \`{a} la d\'{e}finition d'une fonction. McIDE
a aussi des avis tr\`{e}s arr\^{e}t\'{e}s sur le code \matlab, et tente de
reconna\^{i}tre des motifs probl\'{e}matiques courants soit d'avertir le
programmeur ou de les \'{e}liminer automatiquement.

McIDE est compos\'{e} de plusieurs composants plus-ou-mains autonomes,
connect\'{e}s par une interface graphique assez mince. Certains de ces
composants existaient d\'{e}j\`{a}, tel qu'un parseur de \matlab fournit par le
projet \mclab, tandis que d'autres forment les contributions de cette
th\`{e}se, comme un m\'{e}chanisme pour d\'{e}couvrir le graphe d'appels
dynamiques de code \matlab, et un outil pour transformer du code de mani\`{e}re
\`{a} pr\'{e}server sa mise en page.

\`{A} travers la mise en oeuvre de McIDE, un th\`{e}me courant est la
d\'{e}pendance sur de l'information r\'{e}colt\'{e}e en cours
d'\'{e}x\'{e}cution, puisque l'information statique est souvant insuffisante si
nous souhaitons supporter le d\'{e}veloppement de code \matlab arbitraires,
incluant ses fonctions plus dynamiques.
