\matlab is a popular dynamic programming language used for scientific and
numerical programming. It has a very large (and growing) user base, especially
in education, research and engineering applications. A key aspect contributing
to the language's appeal is its accessibility; features like a read-eval-print
loop, dynamic typing, compact and familiar syntax for manipulating arrays and
matrices, easy plotting, access to efficient libraries for many problem domains
and extensive online documentation make \matlab a good language for
prototyping.

Despite all this, \matlab has its warts. Initially conceived as a simple way
for students to use \fortran linear algebra libraries without having to learn
\fortran\footnote{\url{www.mathworks.com/company/newsletters/articles/the-origins-of-matlab.html}},
the language has grown in complexity over the years, with more and more
features bolted on. And with only a black box proprietary reference
implementation in lieu of any sort of language specification, the semantics of
the language can often be inscrutable, particularly around edge cases.
\matlab's aforementioned accessibility is also a double-edged sword, as the
typical \matlab programmer is apt not to have much of a background in computer
science or professional software development, so that large swaths of \matlab
code available online, either in the form of example code or libraries, are not
of a very high quality.

This thesis reports on the design and implementation of McIDE, an integrated
development environment implemented on top of infrastructure provided by the
\mclab compiler toolkit. In addition to providing many traditional IDE features
such as easy code navigation and support for refactorings, McIDE also aims to
improve the state of \matlab code in the wild, be it in terms of performance,
complexity, or amenableness to static analysis, by recognizing common
anti-patterns in \matlab code and warning about them, or automatically
eliminating them in some cases.

\section{Contributions}

The major contributions of this thesis are as follows.

\begin{description}

\item[A mechanism for computing a dynamic call graph of \matlab code:] Many
traditional code navigation features provided by IDEs, such as "jump to
declaration" or "find callers", rely on call graph information. However,
\matlab's semantics make it difficult to statically compute a program's call
graph. We present a dynamic profiling approach to measure a \matlab program's
call graph, and describe some optimizations we implemented to reduce the
overhead of the instrumentation required for the profiling.

\item[A technique for carrying out layout preserving code transformations:] One
of the most useful features commonly provided by IDEs is automated refactoring
support. The most natural and straightforward way to implement a refactoring is
as a tree transformation on a program's abstract syntax tree. However, such
transformations are lossy, as the textual layout of the source code, including
comments, indentation and other whitespace, are lost in the process. We report
on the design and implementation of a library which enables arbitrary source
code transformations to be specified at the AST level, while the underlying
machinery transparently takes care of preserving the layout of the affected
text.

\item[A study of the usage of \matlab's dynamic features in the wild:] \matlab
supports many highly dynamic features, such as the \code{eval} family of
functions, which complicate static analysis, harm performance, and often make
code harder to reason about. We describe the semantics of these features in
detail, and report the results of a study of a large corpus of \matlab code,
investigating the usage patterns of these features.

\item[Techniques for eliminating dynamic features:] Exploiting the findings of
the aforementioned study, we present some techniques for automatically
transforming code away from using these features in some common cases.

\item[An open implementation:] McIDE is developed fully in the open, on top of
  the open source \mclab compiler toolkit for \matlab. Some of the reusable
  infrastructure pieces, such as the layout preserving transformation engine,
  are implemented directly as part of the toolkit, while the IDE itself is
  available as a separate open source
  project\footnote{\url{www.sable.mcgill.ca/mclab/projects/mcide/}}.

\end{description}

\section{Thesis outline}

The rest of the thesis is structured as follows.
\chapref{chap:BackgroundAndOverview} provides some necessary background
information and describes the overall structure of our IDE. The next five
chapters comprise four relatively independent deep dives into its different
components.

\begin{itemize}

\item \chapref{chap:DynamicCallGraphConstruction} explains our approach to
computing a dynamic call graph for \matlab programs, which is used to power the
IDE's code navigation features.

\item \chapref{chap:LayoutPreservingRefactorings} describes our layout
preserving program transformation library, which is used to implement the
mechanics of the refactoring transformations supported by McIDE.

\item \chapref{chap:DynamicFeatureSurvey} presents our investigation into the
usage of \matlab's dynamic features, and
\chapref{chap:DynamicFeatureElimination} follows up with techniques to
automatically eliminate uses of these features where simpler alternatives
exist.

\end{itemize}

Finally, \chapref{chap:Conclusions} concludes the thesis and describes some
opportunities for future work.
