\section{McLab toolkit}

The \mclab toolkit is a collection of useful tools and infrastructure for
dealing with \matlab code. It includes a \matlab parser, and an intraprocedural
static analysis framework with some useful foundational analyses provided out
of the box, such as reaching definitions and the kind analysis for \matlab
\cite{KindAnalysis}.

It actually includes much more, including call graph construction and
sophisticated type and shape inference. However, much of this prior work was
motivated by the ultimate goal of compiling \matlab to a static language such
as \fortran or X10. In this context, it was considered acceptable to carve out
a reasonable subset of \matlab code and rule out any code considered too
dynamic or "wild" to map cleanly onto static semantics. Thus, many of the more
sophisticated analyses assume that many of the features of \matlab that are
difficult to handle simply don't occur.

Since we wish to support the development of arbitrary \matlab code, many of
these assumptions don't hold for us, leaving many components of the toolkit
off-limits for us. A related issue is that since we wish to manipulate and
reason about \matlab source code, we find ourselves working directly with
high-level ASTs, eschewing simplifications or lower-level IRs.

\section{Overall design}

McIDE wires together many independent components into a coherent whole. While
it runs locally on the user's computer, its interface is browser-based, and
largely centered around an instance of the Ace
editor\footnote{\url{http://ace.c9.io/}}, a well-known open source embeddable
text editor component. This browser-based interface contains almost no
important logic; instead, it reacts to user actions by sending HTTP requests to
a server process, which then dispatches the work to the appropriate component.
These components, such as the parser, static analyzers, automated refactorers,
and so on, are all implemented as separate standalone tools, which simply
accept input and produce output. The dispatcher orchestrates these components,
typically by spawning them as child processes and monitoring their standard
output and error streams (shelling out to them, in Unix parlance), although
other means of interprocess communication would also work. This can be seen as
a kind of service-oriented architecture.

There are several advantages to structuring McIDE in this way. For one thing,
it removes the need to grapple with a large monolithic codebase. The different
components can be developed and maintained in isolation, and also reused in
different ways, for example as command line utilities, or as text editor
plugins. Arbitrary, pre-existing components can also be integrated into McIDE,
with only a little effort required to wrap them in a suitable interface. For
example, various bits of functionality, such as the parser, are provided by
pre-existing components of the \mclab toolkit, and exposed to McIDE via simple
shell script wrappers.

The browser-based interface is harder to justify. It is not really a given that
implementing the interface using HTML and JavaScript is preferable to writing a
traditional native desktop application using one of the popular cross-platform
UI toolkits. The main reason is that a natural next step is to generalize McIDE
to be a proper web application -- a "cloud" IDE, accessible over the Internet --
and that less engineering effort would be required to port it to that setting.
The client-server separation also enforces in some sense that the interface
logic be decoupled from the domain logic.

The remainder of this section shows some specific examples of how different
components are integrated at a high-level.

\subsection{Syntax checking and static analysis}

One of McIDE's basic features is on-the-fly syntax checking. As the user types,
the contents of the editor are periodically sent off to the server as a "parse"
request. Upon receiving this request, the server spawns the \mclab toolkit's
\matlab parser, ultimately returning to the frontend either a serialized AST
which can be used as input to other components, or a list of syntax errors,
each with associated line and column location information, to be overlaid in
the margins of the editor.

This workflow is sketched out in \figref{fig:ParserWorkflow}. At the top, the
user has just finished typing in some code. The client sends an HTTP request to
the server with the code as the request body. The server calls the parser
wrapper script, passing in the name of a temporary file containing the code.
The script prints some errors on standard output, and the server uses these to
build a JSON response that it returns to the client, which uses it to display
an error marker at the offending line, with the error messages displayed on
mouse over.

\begin{figure}
\centering
\includegraphics[scale=0.5]{figures/parserworkflow.eps}
\caption{Example of the communication involved to implement on-the-fly syntax
checking.}
\label{fig:ParserWorkflow}
\end{figure}

\subsection{Refactoring}

McIDE supports many automated refactorings, such as Extract Function or Inline
Variable. A refactoring can be viewed as a function taking as input some code
and a user selection (e.g. a highlighted region) and returning either the
transformed code or an error. This fits nicely into our model. When the user
selects some code and selects a refactoring action from a menu, the frontend
sends the project path, the selection, and the choice of refactoring off to the
server as a "refactor" request. (It sends project paths rather than sending the
code directly in case the refactoring affects multiple files). This request is
dispatched to the appropriate refactoring tool, which responds either with a
list of errors, or with a mapping from affected file names to new contents.

This workflow is sketched out in \figref{fig:RefactorWorkflow}. The user has
highlighted an assignment statement and selected the "Inline Variable" action
from the context menu. The client sends an HTTP request to the server with the
choice of refactoring, the active file, and the selection (using a format like
\code{<start-line>,<start-column>-<end-line>,<end-column>}). The server
forwards the request to the inline variable tool. In this case, the wrapper
script happens to write a response of the appropriate format to standard
output, so the server just forwards it along back to the client, which updates
the state of the editor to reflect the changes.

\begin{figure}
\centering
\includegraphics[scale=0.5]{figures/refactorworkflow.eps}
\caption{Example of the communication involved to implement refactorings.}
\label{fig:RefactorWorkflow}
\end{figure}

Beyond the big picture communication here, the actual mechanics of carrying out
these refactorings are explored more deeply in
\chapref{chap:LayoutPreservingRefactorings}.

\subsection{\matlab shell}

McIDE features a \matlab shell, which is implemented by interacting directly
with a running \matlab or Octave instance. The actual communication is largely
implemented by an external library, which works by spawning off a server
written in \matlab, and sending code to it over a message queue. The server
executes the given code via \code{eval}, and responds with a message including
any output (including paths to figures, which are transparently saved to the
filesystem) or errors produced by the execution.

When the user starts working on a project, such a \matlab server is initialized
in the background, and a command prompt is presented to the user alongside the
editor. Any commands entered are sent off to the dispatching server as a "shell
command" request, which forwards them along to the \matlab server, and returns
the results back to the frontend, which can display output and error messages
in its shell, and open new browser windows or tabs to display figures.

McIDE also interacts with the \matlab server behind the scenes for various
reasons.

\begin{itemize}

\item When a function is called, \matlab loads the function's code from the
filesystem and caches the function. This cache is refreshed each time the
command prompt is shown, so that if a function is modified, \matlab will notice
that the last modified timestamp has changed and reload the function. Due to
the nature of our communication with the \matlab server, this refresh mechanism
doesn't happen. For situations like these, \matlab provides the \code{rehash}
builtin function to force the caches to be refreshed. McIDE therefore prepends
a call to \code{rehash} to every command entered by the user before sending it
off.

\item After each command entered by the user, McIDE appends a call to the
\code{save} \matlab builtin function in order to store the state of the
interpreter session to a hidden file associated with the project. When the same
project is loaded later, this session file is loaded via the builtin
\code{load} function, so that all the variables are preserved. A nice bonus is
that the format of the files produced by \code{save} and loaded by \code{load}
are compatible across \matlab and Octave, so the backend can be changed in the
settings menu without losing any active sessions.

\item The proprietary \matlab implementation provides a workspace browser, a
graphical window allowing the user to view (and modify) all the variables in
the current workspace. To replicate this functionality, McIDE makes use of the
\matlab builtin function \code{whos}, which lists all the variables in scope.

\end{itemize}

The overall workflow is sketched out in \figref{fig:ShellWorkflow}. The user
types in a call to the function \code{test}, which is the one being edited, and
presses the return key. The client sends an HTTP request to the server with the
code to run. The server wraps the user's code with some code of its own as
described above, and sends the execution request to an instance of the \matlab
server, which has been pre-initialized to execute in the project's workspace.
The \matlab server responds with information about whether the code triggered
any errors (here the request was successful), what the output was (here "2"
surrounded by some whitespace), and whether there were any figures (not in this
case). The server sends this response back to the client, which displays the
result of the command in the shell.

\begin{figure}
\centering
\includegraphics[scale=0.5]{figures/shellworkflow.eps}
\caption{Example of the communication involved to implement the \matlab shell.}
\label{fig:ShellWorkflow}
\end{figure}

\subsection{Profiling}

A big theme of McIDE's implementation is the reliance on runtime information,
since precise static information is often difficult to come by if we wish to
handle arbitrary \matlab code. When a project is created or imported, McIDE
automatically creates a special file called \code{mcide_entry_point.m}, in
which the user is asked to implement a function that exercises as much of the
project's code as possible. Periodically, and also in response to certain user
actions, a profiling run is triggered, in the form of a "profile" request sent
off to the server. In response, an instrumented version of the project is
created via a source-to-source transformation, and the entry point function is
invoked (via the same mechanism used to implement the \matlab shell) to gather
runtime information.

Currently, the main user of this mechanism is the the dynamic call graph
generator described in \chapref{chap:DynamicCallGraphConstruction}, where the
profiling run records the targets of each call site, but the same mechanism
could be used to gather, for instance, runtime types, or performance profiling
information.
