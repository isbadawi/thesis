\section{\mclab toolkit}

\section{Potential challenges}

The challenges faced in implementing McIDE stem largely from our desire to
support the development of arbitrary \matlab code. While the \mclab toolkit
provides a lot of useful infrastructure, including a working \matlab parser (a
nontrivial accomplishment) and static analysis framework, a lot of prior work
was motivated by the ultimate goal of compiling \matlab to a static language
such as \fortran or X10. In this context, it was considered acceptable to
simply carve out a reasonable subset of \matlab code and rule out any code
considered too dynamic or "wild" to map cleanly onto static semantics. The more
sophisticated static analyses provided by the \mclab toolkit, which include
type and shape inference as well as call graph construction, all rely on the
assumption that the features of \matlab that are difficult to handle simply
don't occur -- an assumption that doesn't hold for us. A related issue is that
since we wish to manipulate and reason about \matlab source code, we find
ourselves working directly with high-level ASTs, eschewing simplifications or
lower-level IRs.

\section{Overall design}

% TODO(isbadawi): end-to-end figure

McIDE wires together many independent components into a coherent whole. While
it runs locally on the user's computer, its interface is browser-based, and
largely centered around an instance of the Ace
editor\footnote{\url{http://ace.c9.io/}}, a well-known open source embeddable
text editor component. This browser-based interface contains almost no
important logic; instead, it reacts to user actions by sending HTTP requests to
a server process, which then dispatches the work to the appropriate component.
These components, such as the parser, static analyzers, automated refactorers,
and so on, are all implemented as separate standalone tools, which simply
accept input and produce output. The dispatcher orchestrates these components,
typically by spawning them as child processes and monitoring their standard
output and error streams (shelling out to them, in Unix parlance), although
other means of interprocess communication would also work. This can be seen as
a kind of service-oriented architecture.

There are several advantages to structuring McIDE in this way. For one thing,
it removes the need to grapple with a large monolithic codebase. The different
components can be developed and maintained in isolation, and also reused in
different ways, for example as command line utilities, or as text editor
plugins. Arbitrary, pre-existing components can also be integrated into McIDE,
with only a little effort required to wrap them in a suitable interface. For
example, various bits of functionality, such as the parser, are provided by
pre-existing components of the \mclab toolkit, and exposed to McIDE via simple
shell script wrappers.

The browser-based interface is harder to justify. It is not really a given that
implementing the interface using HTML and JavaScript is preferable to writing a
traditional native desktop application using one of the popular cross-platform
UI toolkits. The main reason is that a natural next step is to generalize McIDE
to be proper web application -- a "cloud" IDE, accessible over the Internet --
and that less engineering effort would be required to port it to that setting.
The client-server separation also enforces in some sense that the interface
logic be decoupled from the domain logic.

The remainder of this section shows some specific examples of how different
components are integrated at a high-level, and the chapter concludes with a
more specific end-to-end example.

\subsection{Syntax checking and static analysis}

One of McIDE's basic features is on-the-fly syntax checking. As the user types,
the contents of the editor are periodically sent off to the server as a "parse"
request. Upon receiving this request, the server spawns the \mclab toolkit's
\matlab parser, ultimately returning to the frontend either a serialized AST
which can be used as input to other components, or a list of syntax errors,
each with associated line and column location information, to be overlaid in
the margins of the editor.

\subsection{Refactoring}

McIDE supports many automated refactorings, such as Extract Function or Inline
Variable. A refactoring can be viewed as a function taking as input some code
and a user selection (e.g. a highlighted region) and returning either the
transformed code or an error. This fits nicely into our model. When the user
selects some code and selects a refactoring action from a menu, the frontend
sends the project path, the selection, and the choice of refactoring off to the
server as a "refactor" request. (It sends project paths rather than sending the
code directly in case the refactoring affects multiple files). This request is
dispatched to the appropriate refactoring tool, which responds either with a
list of errors, or with a mapping from affected file names to new contents.

Beyond the big picture communication here, the actual mechanics of carrying out
these refactorings are explored more deeply in
\chapref{chap:LayoutPreservingRefactorings}.

\subsection{\matlab shell}

McIDE features a \matlab shell, which is implemented by interacting directly
with a running \matlab or Octave instance. The actual communication is largely
implemented by an external library, which works by spawning off a server
written in \matlab, and sending code to it over a message queue. The server
executes the given code via \code{eval}, and responds with a message including
any output (including paths to figures, which are transparently saved to the
filesystem) or errors produced by the execution.

When the user starts working on a project, such a \matlab server is initialized
in the background, and a command prompt is presented to the user alongside the
editor. Any commands entered are sent off to the dispatching server as a "shell
command" request, which forwards them along to the \matlab server, and returns
the results back to the frontend, which can display output and error messages
in its shell, and open new browser windows or tabs to display figures.

McIDE also interacts with the \matlab server behind the scenes for various
reasons.

\begin{itemize}

\item When a function is called, \matlab loads the function's code from the
filesystem and caches the function. This cache is refreshed each time the
command prompt is shown, so that if a function is modified, \matlab will notice
that the last modified timestamp has changed and reload the function. Due to
the nature of our communication with the \matlab server, this refresh mechanism
doesn't happen. For situations like these, \matlab provides the \code{rehash}
builtin function to force the caches to be refreshed. McIDE therefore prepends
a call to \code{rehash} to every command entered by the user before sending it
off.

\item After each command entered by the user, McIDE appends a call to the
\code{save} \matlab builtin function in order to store the state of the
interpreter session to a hidden file associated with the project. When the same
project is loaded later, this session file is loaded via the builtin
\code{load} function, so that all the variables are preserved. A nice bonus is
that the format of the files produced by \code{save} and loaded by \code{load}
are compatible across \matlab and Octave, so the backend can be changed in the
settings menu without losing any active sessions.

\item The proprietary \matlab implementation provides a workspace browser, a
graphical window allowing the user to view (and modify) all the variables in
the current workspace. To replicate this functionality, McIDE makes use of the
\matlab builtin function \code{whos}, which lists all the variables in scope.

\end{itemize}

\subsection{Profiling}

A big theme of McIDE's implementation is the reliance on runtime information,
since precise static information is often difficult to come by if we wish to
handle arbitrary \matlab code. A few different components, such as the call
graph generator described in \chapref{chap:DynamicCallGraphConstruction}, work
by executing instrumented versions of the project code.

When a project is created or imported, McIDE automatically creates a special
file called \code{mcide_entry_point.m}, in which the user is asked to implement
a function that exercises as much of the project's code as possible.
Periodically, and also in response to certain user actions, a profiling run is
triggered, in the form of a "profile" request sent off to the server. In
response, an instrumented version of the project is created via a
source-to-source transformation, and the entry point function is invoked (via
the same mechanism used to implement the \matlab shell) to gather various bits
of information, including call targets, variable types, and more.
