A refactoring is a code transformation that changes the structure of the code
while preserving its semantics.

Once all the issues surrounding correctness, analyses, preconditions and so on
are sorted out, then from an implementation perspective, a refactoring is most
naturally thought of as a transformation on abstract syntax trees -- for
instance, a refactoring like "extract method" can be boiled down to steps like
"move this statement from this function to that one", "synthesize a new function
call and insert it at this position in the function's body", and so on. Given this,
a natural structure for a refactoring tool consists of parsing code, then performing
transformations on its AST, and then pretty printing the transformed AST to retrieve
source code to present to the programmer. Many refactoring tools -- particularly
ones developed for research purposes -- operate this way.

The problem with such approaches is that by construction the AST does not contain
enough information to accurately reconstruct the input source code. Typically
among the casualties are whitespace, comments and syntactic sugar. If a particular
construct is not sensitive to order, then order is frequently lost. For example,
in MATLAB, nested functions can appear anywhere inside a function but their position
and order don't matter. In the McLab AST, is it not straightforward to get at the
program point at which a nested function was declared, and the pretty printer simply
moves all nested functions to the bottom of their enclosing function.

\figref{Fig:LostLayout} shows a MATLAB program and the result of parsing and then
pretty-printing it using the McLab toolkit. The two programs are behaviorally
equivalent, but contain many syntactic differences.

\begin{figure}
\begin{minipage}{0.5\linewidth}
\begin{lstlisting}[numbers=none]
% cube takes a number x and returns its cube.
function y = cube(x)
  % This is a nested function that computes the square of a value.
  function v = square(u)
    v = u * u;
  end
  y = x * square(x);
end
\end{lstlisting}
\end{minipage}
\hfill
\hspace{.3cm}
\hfill
\begin{minipage}{0.5\linewidth}
\begin{lstlisting}[numbers=none]
function [y] = cube(x)
% cube takes a number x and returns its cube.
% This is a nested function that computes the square of a value.
  y = (x * square(x));
  function [v] = square(u)
    v = (u * u);
  end
end
\end{lstlisting}
\end{minipage}
\caption{An example of the lossy parsing and unparsing roundtrip.}
\label{Fig:LostLayout}
\end{figure}

Users of an automated refactoring tool are unlikely to be accepting of such invasive
changes to a program's text. As such, it is important for the tool to be aware of
the layout of the program when performing refactorings. For a given AST
transformation, it should endeavor to perform the minimal textual changes needed
to reflect the transformation in the program text. In particular, unaffected
portions of the program should not undergo any textual changes.

Despite this, it is still convenient to express refactorings as tree
transformations. Refactorings would be much harder to implement and maintain if
they had to be expressed as textual transformations, or as a mixture of tree and
text transformations that had to be kept in sync.

Our goal is to be able to implement refactorings purely as tree transformations,
and to have minimal textual changes automatically computed from them. In order
to accomplish this, we introduce a simple transformation API, which exposes a
small set of tree manipulation operations. Instead of directly manipulating AST
nodes, refactorings are implemented in terms of this API. Behind the scenes, the
implementation of the API includes logic that keeps track the input program text
in addition to the AST, and keeps the two in sync.

-------

Related work (TODO flesh this out)

The work that most closely resembles ours is HaRe, a refactoring tool for Haskell.
It uses a similar approach of synchronizing an AST with a token stream in order
to pretty print refactored programs.

Waddington and Yao [LDTA '05] tackled the same problem, which termed "the problem
of style disruption", with Proteus, their refactoring tool for C and C++. Their
approach to use a specialized AST called a "Literal-Layout AST (LL-AST)", where
literals token and whitespace nodes are interspersed alongside the regular nodes.

* RefactorErl
* That Scala refactoring guy

The Eclipse JDT contains infrastructure for modifying code at two levels --
a lower-level API for describing text manipulation primitives, and a higher-level
AST rewriting API, which accepts descriptions of changes to AST nodes and uses the
text manipulation API to try and perform the textual changes required to represent
the AST changes. The approach is similar to ours in spirit; one big difference is
that since that our approach is implemented largely a standalone tool, we rely
solely on lexing and parsing as primitives, while Eclipse's implementation
benefits from more sophisticated integration with a scriptable text editor.
