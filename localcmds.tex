\lstset{
  language=Java,
  basicstyle=\ttfamily\footnotesize,
  keywordstyle=\bfseries\color[rgb]{.498,.000,.333},
  identifierstyle=,
  commentstyle=\color[rgb]{.753,.753,.753},
  stringstyle=\color[rgb]{.164,.000,1.00},
  showstringspaces=false,
  tabsize=2,
  columns=fullflexible,
  frame=single,
  framerule=0.6pt,
  backgroundcolor=\color{white},
  rulecolor=\color[rgb]{0.843,0.843,0.843},
  captionpos=b,
  numbers=left,
  numberstyle=\scriptsize\color[rgb]{.501,.501,.501},
  stepnumber=1,
  breaklines=true,
  breakatwhitespace=true
}

\newcommand{\code}[1]{{\lstinline{#1}}}
\newcommand{\codekeyword}[1]{\textbf{\code{#1}}}

\newcommand{\chapref}[1]{\textit{Chapter \ref{#1}}}
\newcommand{\appendixref}[1]{\textit{Appendix \ref{#1}}}
\newcommand{\sectionref}[1]{\textit{Section \ref{#1}}}
\newcommand{\figref}[1]{\textit{Figure \ref{#1}}}
\newcommand{\tableref}[1]{\textit{Table \ref{#1}}}
\newcommand{\lstref}[1]{\textit{Listing \ref{#1}}}
\newcommand{\lstrefTwo}[2]{\textit{Listings \ref{#1} \& \ref{#2}}}
\newcommand{\lstrefN}[2]{\textit{Listings \ref{#1} - \ref{#2}}}
\newcommand{\secref}[1]{Sec.~\ref{#1}}
\newcommand{\figureref}[1]{Figure~\ref{#1}}
\newcommand{\equationref}[1]{Equation~\ref{#1}}
\newcommand{\eqnref}[1]{(\ref{#1})}
