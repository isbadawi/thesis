 %% Copyright (C) 2006 Ahmer Ahmedani

\documentclass[MSc,twoside,openright]{Thesis}

\newif\ifdraft
% \drafttrue

%== Preamble ==================================================================

\usepackage[french]{babel}
\usepackage[T1]{fontenc}
\usepackage[utf8]{inputenc}

\usepackage{placeins}
\usepackage{tabularx}
\usepackage{multicol}
\usepackage{xcolor}    % Keyword highlighting in listing
\usepackage{listings}  % Typeset source code listings,
                       % Files in current direcotry
                       % ( listings.cfg listings.sty lstdoc.sty lstlang1.sty
                       % lstlang2.sty lstlang3.sty lstmisc.sty ) are listings
                       % version 1.4, should not be removed, 1.3 version cause
                       % problem in the left line of the frame (standalone use
                       % of 1.3 will not cause this problem, but in this
                       % project, it does)
\usepackage[bw]{mcode} % mcode listings

\usepackage{ifthen}    % For conditional commands
\usepackage{ifpdf}     % Provide \ifpdf conditional
\usepackage{xspace}    % Define commands that don't eat spaces
\usepackage{type1cm}
\usepackage{times}     % Use Times *deprecated*
%% listings.sty doesn't seem to pretty print code listings if the
%% `times' packages is not loaded. Why? Who knows. It will do it fine
%% in a simple document, just not in this one.
\usepackage{mathptmx}  % Use Times for roman family and math
% \usepackage{mathpazo}  % Palantino
% \usepackage{chancery}
% \usepackage{bookman}
% \usepackage{newcent}
% \usepackage{charter}
\usepackage[scaled]{helvet}    % Use Helvetica for sans serif family
%\usepackage{avant}     % Use Avant Garde for sans serif family
\usepackage{pifont}    % Symbol and Zapf Dingbats
%% TODO: investigate fourier package (Adobe Utopia fonts)

\usepackage{fancyhdr}  % Fancy page headers
\usepackage{verbatim}  % provide comment environments
\usepackage{fancyvrb}  % improved verbatim and verbatim* environments

%\usepackage{hyperref} % split urls
\usepackage{url}       % For nicely formatted URLs


%% Nicer formatting of figure captions:
\usepackage[format=hang,font={small,sf},labelfont=bf,labelsep=space]{caption}
%\usepackage[tight]{subfigure} % subfigures. replace with subfig?
\usepackage{subfig}
\usepackage{setspace}
\usepackage{longtable} % Make tables span multiple pages
\usepackage{multirow}  % Table cells that span multiple rows
\usepackage{dcolumn}   % Line up decimal sep in tabular columns
% \usepackage{warpcol}   % Alternate to dcolumn
\usepackage{color}     % Allows text and page background colors to be set
\usepackage{colortbl}  % Coloured tables
\usepackage[final]{graphicx}  % Better support for graphics
\usepackage{layout}    % produces a figure that describes the page layout
\usepackage{titlesec}  % to redefine typesetting of \paragraph
\usepackage{rotating}  % for rotated table headings
% Note: yap does not support rotating, so convert .dvi to .pdf and then
%    preview the .pdf file
% for algorithms
\usepackage[algo2e, algochapter, ruled, linesnumbered, lined]{algorithm2e}
%% Make sure that the bibliography is listed in the table of contents,
%% but that the table of contents itself is not.
% XXX: doesn't seem to work
%\usepackage[nottoc]{tocbibind}
\usepackage[none]{tocbibind}
%\usepackage{hyphenat} %enhanced hyphenation,
%\usepackage[htt]{hyphenat} %htt enables hyphenation of text typeset
% some better colours for hyperref links:
\definecolor{darkgreen}{rgb}{0.2,0.5,0.1}
\definecolor{darkblue} {rgb}{0.1,0.4,0.5}
\definecolor{maroon}   {rgb}{0.45,0.05,0.25}
\definecolor{red}      {rgb}{1,0,0}
\ifpdf
  %% TODO: can I use variables here for name, title, etc?
  \usepackage[
    pdftex,
    colorlinks=true,
    linkcolor=maroon,
    citecolor=darkgreen,
    pagecolor=maroon,
    urlcolor=darkblue,
    pdftitle={The MetaLexer Lexer Specification Language},
    pdfauthor={Andrew Casey},
    pdfsubject={The MetaLexer Lexer Specification Language},
    pdfkeywords={MetaLexer, Lexer, Scanner, Extensible, Modular}
  ]
  {hyperref} % hyper-text links, etc.
\else
  \usepackage[
    dvips,
    breaklinks=true,
    colorlinks=true,
    linkcolor=maroon,
    citecolor=darkgreen,
    pagecolor=maroon,
    urlcolor=darkblue,
  ]
  {hyperref}
\fi


% Use the ams math packages
\usepackage{amssymb,amsmath}
\usepackage{bnf}

% -- Customize Layout ---------------------------------------------------------

% custom page headers:

\lhead[]{\fancyplain{}{\nouppercase{\rightmark}}}
\rhead[\fancyplain{}{\nouppercase{\leftmark}}]{}
\addtolength{\headwidth}{10mm} % => extend line out into margin

%\fancyhead[EL]{THESIS DRAFT}
%\fancyhead[OR]{THESIS DRAFT}


\titleformat{\paragraph}[hang]{\normalfont\it}{}{0em}{}

% Make LaTeX relax a little wrt figure placement
\renewcommand{\topfraction}{0.85}
\renewcommand{\textfraction}{0.1}
\renewcommand{\floatpagefraction}{0.75} % Prevent half-empty pages

% Tell LaTeX to not "bottom justify" text. This prevents ugly
% spaces between paragraphs in columns when LaTeX stretches them.
\raggedbottom

% Set the depth for the table of contents to 2 for non-draft output
\ifdraft
\else
\setcounter{tocdepth}{2}
\fi

% Set the value of the margin of all algorithms.
% The default value is \leftskip plus \parindent
%  when the algorithm2e package is loaded.
\incmargin{\parindent} %increase one more \parindent to the default
% Set font of comment in algorithms
\newcommand{\algcommentfont}[1]{{\small \texttt{#1}}}
\SetCommentSty{algcommentfont}

%----------Matlab---------------------
\newcommand{\abc}{\textsl{abc}\xspace}
\newcommand{\amc}{\textsl{amc}\xspace}
\newcommand{\matlab}{{\sc Matlab}\xspace}
\newcommand{\smatlab}{{\sc Matlab}}
\newcommand{\smclab}{\textrm{\textsl{Mc}\textbf{\textsc{Lab}}}}
\newcommand{\mclab}{\smclab\xspace}
\newcommand{\mcirs}{\textrm{\textsl{Mc}\textbf{\textsc{ir}}}}
\newcommand{\smcir}{\mcirs}
\newcommand{\mcir}{\smcir\xspace}
\newcommand{\mcasts}{\textrm{\textsl{Mc}\textbf{\textsc{ast}}}}
\newcommand{\smcast}{\mcasts}
\newcommand{\mcast}{\smcast\xspace}
\newcommand{\smcjit}{\textrm{\textsl{Mc}\textbf{\textsc{jit}}}}
\newcommand{\mcjit}{\smcjit\xspace}
\newcommand{\java}{\textsc{Java}\xspace}
\newcommand{\sjava}{\textsc{Java}}
\newcommand{\fortran}{\textsc{Fortran}\xspace}
\newcommand{\xten}{{\sc X10}\xspace}
\newcommand{\mcbench}{{\sc McBench}\xspace}
\newcommand{\mcfor}{{\sc McFor}\xspace}
\newcommand{\mcsaf}{{\sc McSaf}\xspace}
\newcommand{\kw}[1]{\texttt{#1}}


\newcommand{\rednote}[1]{#1} %{\textcolor{red}{#1}}
\newcommand{\mynote}[1]{} %{\marginpar{\scriptsize{\rednote{#1}}}}

% MATLAB lang. def. for listings
\lstdefinelanguage{MATLAB}{
    sensitive=true, % Case sensitive identifiers
    morecomment=[l]{\%}, % Line-based comment character
    morestring=[b]', % String character
    morekeywords= {
		function,
		for,
		while,
		if,
		else,
		elseif,
		end,
		aspect,
		patterns,
		actions,
		methods,
		properties,
		class,
		classdef,
		script,
		loops,
		set,
		get,
		call,
		execution,
		mainexecution,
		loop,
		loopbody,
		loophead,
		within,
		before,
		after,
		around
	},
	commentstyle=\color[rgb]{.600,.600,.600}, % grey comments
}

% -- Input local commands and hyphenation rules -------------------------------
% -- Custom Environments ---
\definecolor{darkgrey} {rgb}{0.843,0.843,0.843}
\definecolor{lightgrey} {rgb}{0.979,0.979,0.979}
\lstset{
        language=[AspectJ]Java, %keyword highlighting seems annoying
        morekeywords={declare, parents},
        basicstyle=\ttfamily\footnotesize, % use fixed-width font
        keywordstyle=\bfseries\color[rgb]{.498,.000,.333}, % eclipse color, bold
        %keywordstyle=\bfseries, % bold keywords
        identifierstyle=,       % nothing happens
        %commentstyle=\color[rgb]{.247,.498,.372}, % eclipse color
        commentstyle=\color[rgb]{.753,.753,.753}, % grey comments
        stringstyle=\color[rgb]{.164,.000,1.00},  % eclipse color
        %stringstyle=\ttfamily,  % typewriter type for strings
        showstringspaces=false, % no special string spaces
	    tabsize=2,
        columns=fullflexible,   % Use flexible column format (for comments)
        frame=single, %
        framerule=0.6pt, %
        backgroundcolor=\color{white}, %
        rulecolor=\color{darkgrey}, %
        captionpos=b, %
	    numbers=left, 
	    numberstyle=\scriptsize\color[rgb]{.501,.501,.501},
	    stepnumber=1,
	    breaklines=true,
	    breakatwhitespace=true
}

\newcommand{\code}[1]{{\small \texttt{#1}}}
\newcommand{\codekeyword}[1]{\textbf{\code{#1}}}

% MetaLexer

\newcommand{\mlkw}[1]{\codekeyword{#1}\xspace}
\newcommand{\ml}[1]{\textit{#1}}
\newcommand{\jflexkw}[1]{\codekeyword{#1}\xspace}
\newcommand{\jflex}[1]{\textit{#1}}
%\newcommand{\java}[1]{\textit{#1}}
\newcommand{\weburl}[1]{\textit{#1}}
\newcommand{\file}[1]{\textit{#1}}
\newcommand{\target}[1]{\textit{#1}}
\newcommand{\property}[1]{\textit{#1}}
\newcommand{\cli}[1]{\textit{#1}}
\newcommand{\chapref}[1]{\textit{Chapter \ref{#1}}}
\newcommand{\appendixref}[1]{\textit{Appendix \ref{#1}}}
\newcommand{\sectionref}[1]{\textit{Section \ref{#1}}}
\newcommand{\figref}[1]{\textit{Figure \ref{#1}}}
\newcommand{\tableref}[1]{\textit{Table \ref{#1}}}
\newcommand{\lstref}[1]{\textit{Listing \ref{#1}}}
\newcommand{\lstrefTwo}[2]{\textit{Listings \ref{#1} \& \ref{#2}}}
\newcommand{\lstrefN}[2]{\textit{Listings \ref{#1} - \ref{#2}}}

\newcommand{\secref}[1]{Sec.~\ref{#1}}
\newcommand{\figureref}[1]{Figure~\ref{#1}}
\newcommand{\equationref}[1]{Equation~\ref{#1}}
\newcommand{\eqnref}[1]{(\ref{#1})}
\newcommand{\RC}{reference-counting-based }



\newcommand{\patANY}{\mlkw{<<ANY>>}}
\newcommand{\patEOF}{\mlkw{<<EOF>>}}
\newcommand{\mpatANY}{\mlkw{<ANY>}}
\newcommand{\mpatBOF}{\mlkw{<BOF>}}

\newcommand{\red}[1]{\textcolor{red}{#1}}
\newcommand{\note}[1]{\textcolor{red}{\textbf{#1}}}
\newcommand{\variation}[2]{\textbf{\textcolor{green}{#1} \red{or} \textcolor{blue}{#2}}}

%\newcommand{\mcode}[1]{\lstinline[language=MATLAB]|#1|}
\newcommand{\jcode}[1]{\lstinline[language=Java]|#1|}
\newcommand{\pcode}[1]{\lstinline[language=pseudo]|#1|}


\newcommand{\into}{ 
   \end{minipage}
   \parbox{1cm}{\LARGE\centering $\mathbf{\Rightarrow}$}
   \begin{minipage}{5cm}
 }
\newenvironment{transform}
{
  \begin{center}
    \begin{minipage}{5cm}
}
{
  \end{minipage}
\end{center}
}
\lstnewenvironment{mtrans}
{
\lstset{language=matlab,numbers=none,frame=single}
}
{}

%%% Local Variables:
%%% mode: LaTeX
%%% TeX-master: "thesis"
%%% End: 


% Make matlab the default language
\lstset{
  language=MATLAB,
  mathescape=true
}

%== Title Information =========================================================

%--------------------- 70 character title limit -----------------------
\title{TITLE TO BE DETERMINED}

\author{Ismail Badawi}

\Department{School of Computer Science}
\Institution{McGill University}
\Location{Montr\'eal}

\SubmitDate{Submission date to be determined}

\CopyrightMessage{Copyright \copyright\ 2014 Ismail Badawi}

%== Document ==================================================================

\begin{document}

\pagestyle{empty}

\maketitle
\cleardoublepage

\preface % -- Front Matter ----------------------------------------------------

\begin{Abstract}

\matlab is a dynamic scientific language used by scientists, engineers and
students worldwide.  Although \matlab is very suitable for rapid prototyping
and development,  \matlab users
often want to convert their final \matlab programs to a static language such as {\sc
FORTRAN}, to integrate them into already existing programs of that language,
to leverage the performance of powerful static compilers, or to 
ease the distribution of executables.

This thesis presents an extensible object-oriented toolkit to help
facilitate the generation of static programs from dynamic \matlab
programs.  Our open source toolkit, called the \matlab Tamer, targets
a large subset of \matlab. Given information about the entry point of
the program, the \matlab Tamer builds a complete callgraph, transforms
every function into a reduced intermediate representation, and
provides typing information to aid the generation of static code.

In order to provide this functionality, we need to handle a large
number of \matlab builtin functions. Part of the Tamer framework is
the builtin framework, an extensible toolkit which provides a
principled approach to handle a large number of builtin functions.  To
build the callgraph, we provide an interprocedural analysis framework,
which can be used to implement full-program analyses.  Using this
interprocedural framework, we have developed value analysis, an extensible
interprocedural analysis to estimate \matlab types, which helps
discover the call edges needed to build the call graph.

In order to make the static analyses even possible, we disallow a
small number of \matlab constructs and features, but attempt to
support as large a subset of \matlab as possible.  Thus, by both
slightly restricting \matlab, and by providing a framework with
powerful analyses and simplifying transformations, we can
``Tame \matlab''.





\end{Abstract}

\begin{Resume}
{\sc Matlab}\textsuperscript{\textregistered} est un langage de programmation
dynamique populaire chez les scientifiques. L'utilisateur typique de \matlab
n'est pas un programmeur professionel; \matlab est
principalement utilis\'{e} par des scientifiques, des ing\'{e}nieurs et des
\'{e}tudiants, et doit sa popularit\'{e} en grande mesure \`{a} sa syntaxe de
haut niveau et \`{a} son \'{e}ventail de librairies pour toutes sortes de
domaines dans les sciences. Le manque d'\'{e}xperience des programmeurs
\matlab, combin\'{e} \`{a} ses s\'{e}mantiques mal sp\'{e}cifi\'{e}es et
souvent contraires \`{a} l'intuition, m\`{e}ne \`{a} du code \matlab qui est
difficile \`{a} comprendre et maintenir.

Dans cette th\`{e}se, nous pr\'{e}sentons McIDE, un environnement de
d\'{e}veloppement int\'{e}gr\'{e} pour \matlab. McIDE fournit des outils visant
\`{a} aider les programmeurs \matlab \`{a} \'{e}crire de meilleurs programmes,
incluant des remaniements automatiques et des fonctions de navigation de code,
par exemple permettant de sauter \`{a} la d\'{e}finition d'une fonction. McIDE
a aussi des avis tr\`{e}s arr\^{e}t\'{e}s sur le code \matlab, et tente de
reconna\^{i}tre des motifs probl\'{e}matiques courants et soit d'avertir le
programmeur ou de les \'{e}liminer automatiquement.

McIDE est compos\'{e} de plusieurs composants plus-ou-mains autonomes,
connect\'{e}s par une interface graphique assez mince. Certains de ces
composants existaient d\'{e}j\`{a}, tel qu'un parseur de \matlab fournit par le
projet \mclab, tandis que d'autres forment les contributions de cette
th\`{e}se, comme un m\'{e}chanisme pour d\'{e}couvrir le graphe d'appels
dynamiques de code \matlab, et un outil pour transformer du code de mani\`{e}re
\`{a} pr\'{e}server sa mise en page.

\`{A} travers la mise en oeuvre de McIDE, un th\`{e}me courant est la
d\'{e}pendance sur de l'information r\'{e}colt\'{e}e en cours
d'\'{e}x\'{e}cution, puisque l'information statique est souvant insuffisante si
nous souhaitons supporter le d\'{e}veloppement de code \matlab arbitraires,
incluant ses fonctions plus dynamiques.

\end{Resume}

\chapter*{Acknowledgements}
I'd like to thank my advisor Laurie Hendren, who displayed a lot of patience,
even when it wasn't clear whether I was making forward progress.

I'd like to thank the Sable lab members and alumni I've interacted with during
the time that I've spent here -- among them Anton Dubrau, Matthieu Dubet, Xu Li,
Vineet Kumar, Rahul Garg, Sameer Jagdale, Faiz Khan, Sujay Kathrotia, Vincent
Foley-Bourgon and Erick Lavoie.

I'd particularly like to thank my friends Jerina Harizaj and Lei Lopez, who
were positive forces in my life during periods where I felt very frustrated and
dispirited.

Finally, I'd like to thank my parents, my brother and my sister for their
lifelong support and encouragement.



\renewcommand{\contentsname}{Table of Contents}%
\addto\captionsenglish{%
  \renewcommand{\contentsname}%
    {Table of Contents}%
}
\addto\captionsenglish{%
  \renewcommand{\lstlistlistingname}%
    {List of Listings}%
}

\tableofcontents
\listoffigures
\listoftables
% Make the 'list of listings' page follow the conventions for the title
\renewcommand{\lstlistlistingname}{List of Listings}
%This line results in a duplicate entry in the .out file
%\renewcommand{\lstlistoflistings}{\begingroup
%  \tocfile{\lstlistlistingname}{lol}
%\endgroup}
%\lstlistoflistings
%\listoflistings
%\listofalgorithmes
\cleardoublepage

\maintext % -- Main Body ------------------------------------------------------

\pagestyle{fancyplain}

\chapter{Introduction} \label{chap:Introduction}
\matlab is a popular dynamic programming language used for scientific and
numerical programming. It has a very large (and growing) user base, especially
in education, research and engineering applications. A key aspect contributing
to the language's appeal is its accessibility; features like a read-eval-print
loop, dynamic typing, compact and familiar syntax for manipulating arrays and
matrices, easy plotting, access to efficient libraries for many problem domains
and extensive online documentation make \matlab a good language for
prototyping.

Despite all this, \matlab has its warts. Initially conceived as a simple way
for students to use \fortran linear algebra libraries without having to learn
\fortran, the language has grown in complexity over the years, with more and
more features bolted on. And with only a black box proprietary reference
implementation in lieu of any sort of language specification, the semantics of
the language can often be inscrutable, particularly around edge cases.
\matlab's aforementioned accessibility is also a double-edged sword, as the
typical \matlab programmer is apt not to have much of a background in computer
science or professional software development, so that large swaths of \matlab
code available online, either in the form of example code or libraries, are not
of a very high quality.

This thesis reports on the design and implementation of McIDE, an integrated
development environment implemented on top of infrastructure provided by the
\mclab compiler toolkit. In addition to providing many traditional IDE features
such as easy code navigation and support for refactorings, McIDE also aims to
improve the state of \matlab code in the wild, be it in terms of performance,
complexity, or amenableness to static analysis, by recognizing common
anti-patterns in \matlab code and warning about them, or automatically
eliminating them in some cases.

\section{Contributions}

The major contributions of this thesis are as follows.

\begin{description}

\item[A mechanism for computing a dynamic call graph of \matlab code:] Many
traditional code navigation features provided by IDEs, such as "jump to
declaration" or "find callers", rely on call graph information. However,
\matlab's semantics make it difficult to statically compute a program's call
graph. We present a dynamic profiling approach to measure a \matlab program's
call graph, and describe a few optimizations we implemented to reduce the
overhead of the instrumentation required for the profiling.

\item[A technique for carrying out layout preserving code transformations:] One
of the most useful features commonly provided by IDEs is automated refactoring
support. The most natural and straightforward way to implement a refactoring is
as a tree transformation on a program's abstract syntax tree. However, such
transformations are lossy, as the textual layout of the source code, including
comments, indentation and other whitespace, are lost in the process. We report
on the design and implementation of a library which enables arbitrary source
code transformations to be specified at the AST level, while the underlying
machinery transparently takes care of preserving the layout of the affected
text.

\item[A study of the usage \matlab's dynamic features in the wild:] \matlab
supports many highly dynamic features, such as the \code{eval} family of
functions, which complicate static analysis, harm performance, and often make
code harder to reason about. We describe the semantics of these features in
detail, and report the results of a study of a large corpus of \matlab code,
investigating the usage patterns of these features.

\item[Techniques for eliminating dynamic features:] Exploiting the findings of
the aforementioned study, we present a few techniques for automatically
transforming code away from using these features in some common cases.

\item[An open implementation:] McIDE is developed fully in the open, on top of
the open source \mclab compiler toolkit for \matlab. Some of the reusable
infrastructure pieces, such as the layout preserving transformation engine, are
implemented directly as part of the toolkit, while the IDE itself is available
as an separate open source project
\footnote{\url{http://www.sable.mcgill.ca/mclab/projects/mcide/}}.

\end{description}

\section{Thesis outline}

The rest of the thesis is structured as follows.
\chapref{chap:BackgroundAndOverview} provides some necessary background
information and describes the overall structure of our IDE. The next five
chapters comprise four relatively independent deep dives into its different
components.

\begin{itemize}

\item \chapref{chap:DynamicCallGraphConstruction} explains our approach to
computing a dynamic call graph for \matlab programs, which is used to power the
IDE's code navigation features.

\item \chapref{chap:LayoutPreservingRefactorings} describes our layout
preserving program transformation library, which is used to implement the
mechanics of the refactoring transformations supported by McIDE.

\item \chapref{chap:DynamicFeatureSurvey} presents our investigation into the
usage of \matlab's dynamic features, and
\chapref{chap:DynamicFeatureElimination} follows up with techniques to
automatically eliminate uses of these features where simpler alternatives
exist.

\item TODO

\end{itemize}

Finally, \chapref{chap:Conclusions} concludes the thesis and describes some
opportunities for future work.


\chapter{Overall system design} \label{chap:OverallSystemDesign}
\input{text/overall-system-design}

\chapter{Dynamic call graph construction} \label{chap:DynamicCallGraphConstruction}
Modern IDEs provide many useful code navigation facilities, for instance
allowing users to jump from a call site to the declaration of the called
function, or to find all the call sites of a particular function definition.
The reliability of such features is contingent on the availability of accurate
call graph information. However, \matlab's dynamic typing and dynamic features
complicate the problem of statically computing a precise call graph.

Previous work on \matlab call graph construction operated on a \matlab subset,
carefully ruling out those features which aren't amenable to static analysis,
with the ultimate goal of compiling \matlab to a statically typed language such
as \fortran or \xten \cite{Tamer}. As we mean to support regular \matlab
development, carving out such a subset is not an acceptable approach.

In this chapter, we present our approach to computing an accurate call graph
for arbitrary \matlab code. Rather than relying on static analysis, we resort
to extracting this information dynamically, by instrumenting the input programs
and tracing their actual execution on a \matlab implementation.

\section{\matlab features complicating call graph computation}

\section{Call graph tracing instrumentation}

Since static analysis of \matlab code is difficult and easily misled in the
presence of dynamic features, we resort to dynamic analysis to extract
precision sufficient for our needs. Denker et al.
\cite{AbstractionsForDynamicAnalysis} identify different approaches available
to dynamic analysis tool developers for gathering runtime data:

\begin{itemize}
\item \emph{Source code modification} and, relatedly, \emph{logging services}.
This is the approach we ultimately use, as we discuss later.
\item \emph{Bytecode modification} or \emph{instrumenting the virtual machine}.
This requires knowledge of the internals of the \matlab virtual machine, and as
the reference \matlab implementation is a proprietary closed-source black box,
this isn't an option for us.
\item \emph{Method wrappers}. This refers to some mechanism for introducing
code to be executed before, after, or instead of a function. Our particular
source-to-source transformation, described later, can be seen of an instance of
this technique.
\item \emph{Debuggers}. While the reference \matlab implementation does include
a debugger, we prefer not to couple ourselves too tightly to it, as it is not
under our control.
\end{itemize}

The most natural and portable approach is source code modification. We can
implement it using the infrastructure provided as part of the \mclab toolkit.

The high-level idea is to insert logging statements before every possible call
site, and at the start of every function or script. After executing the
transformed code, we can post-process the logs and match up call sites with
their targets, since the target will follow the call in the log. We define a
unique identifier $identifier(n)$ for every call site and call target $n$; this
consists of the name of $n$ (the variable name if it's a variable, the function
name if it's a function definition, the script name if it's a script, and the
string \texttt{<lambda>} if it's an anonymous function expression), the file
it's contained in and its position (line and column) within that file. This
format comes in handy when it comes to implementing navigation features in an
IDE, as these typically take a textual range (e.g. a mouse selection) as input.

The transformation depends on a few functions (listed in
\figref{Fig:CallgraphRuntime}) being available at runtime. The
\texttt{mclab\_callgraph\_init} and \texttt{mclab\_callgraph\_log} functions
are straightforward; the former takes a path to a log file, creates it and
makes a handle to it globally accessible, while the latter takes a string and
writes it to the file. \newline \texttt{mclab\_callgraph\_log\_then\_run} is more
complicated; it takes a string, a variable (which is possibly a function
handle) and a variable number of arguments. If the given variable is a function
handle (either a function handle expression, or a variable that contains a
function handle), then we log the string to the file, and in either case we
forward the arguments to the variable.

\begin{figure}[htbp]
\lstinputlisting[numbers=none]{code/callgraph_runtime.m}
\caption{The runtime components of the callgraph tracer.}
\label{Fig:CallgraphRuntime}
\end{figure}

Assuming these runtime functions are available, we traverse the whole project
and perform the following transformations.

\begin{itemize}

\item For every function or script $f$, we insert a call to
  \texttt{mclab\_callgraph\_log} as the first statement, passing the string
  \texttt{enter} followed by $identifier(f)$.

\item For every anonymous function definition $f$, we replace the body $b$ of the
  anonymous function with a call to \texttt{mclab\_callgraph\_log\_then\_run},
  passing the string \texttt{enter} followed by $identifier(f)$ as the first
  argument, and the original expression $b$ wrapped in an anonymous function
  expression taking no arguments as the second argument.

\item We replace every function call $n$ (as identified by the kind analysis)
  with a call to \texttt{mclab\_callgraph\_log\_then\_run}, passing the string
  \texttt{call} followed by $identifier(n)$ as the first argument, a handle to
  the target function as the second argument, and copies of the original
  arguments as the rest of the arguments.

  One caveat here is that there can be functions whose return value
  depends on the current execution context. For instance, \texttt{nargin} and
  \texttt{nargout} are builtin functions that return the number of input and
  output parameters passed to the current function. If we call these functions
  inside \texttt{mclab\_callgraph\_log\_then\_run} instead of the original
  function, they won't necessarily return the same value. As such, we avoid
  instrumenting calls to these functions, among other reflective functions such
  as \texttt{narginchk} and \texttt{inputname}.

\item While the kind analysis distinguishes between function calls and variable
  accesses, it doesn't distinguish among the latter between array accesses and
  function handle invocations. In order to accurately trace control flow through
  function handles, we also instrument variable accesses in the same way as for
  function calls, only rather than passing in a function handle expression as
  the second argument, we just pass in the variable. At runtime,
  \texttt{mclab\_callgraph\_log\_then\_run} makes use of \matlab's reflective
  features to identify function handles, and only logs the call event in those
  cases. One small detail here is that an array access might have a colon
  literal as one of its arguments, and passing it to a function instead will
  cause \matlab to generate an error at runtime. In order for the
  transformation to be correct, we go through and replace any colon literals
  with colon string literals.

\end{itemize}

Finally, in order to trigger a tracing execution, an entry point is needed --
that is, a piece of code that will attempt to exercise as much of the subject
code as possible. This is handed off to the tracing machinery, which will first
instrument the project as described (in a temporary folder), create a temporary
file to hold the trace, and invoke \matlab, first calling
\texttt{mclab\_callgraph\_init} with the path to the log file, and then the
entry point. Once the execution is over, the trace is processed, and call
graph edges are identified by looking for \texttt{call} events that are
immediately followed by an \texttt{enter} event. \figref{Fig:CallgraphBefore},
\figref{Fig:CallgraphAfter}, \figref{Fig:CallgraphTrace} and \figref{Fig:Callgraph}
together show an end-to-end example; two \matlab files, the result of
instrumenting them, the generated call trace, and the call graph constructed
from it.

\begin{figure}[htbp]
\begin{minipage}{\linewidth}
  \lstinputlisting[numbers=none]{code/callgraph-example/before/for_each_file.m}
\end{minipage}
\begin{minipage}{\linewidth}
  \lstinputlisting[numbers=none]{code/callgraph-example/before/code_size.m}
\end{minipage}
\caption{The application code; for\_each\_file recursively traverses a directory
tree and invokes a handler for each file with the given extension. code\_size
uses it to add up the sizes of all the m-files in the current directory.}
\label{Fig:CallgraphBefore}
\end{figure}

\begin{figure}[htbp]
\begin{minipage}{\linewidth}
  \lstinputlisting[numbers=none]{code/callgraph-example/after/for_each_file.m}
\end{minipage}
\begin{minipage}{\linewidth}
  \lstinputlisting[numbers=none]{code/callgraph-example/after/code_size.m}
\end{minipage}
\caption{The same code after instrumentation (with the \texttt{mclab\_callgraph}
prefix omitted for brevity.)}
\label{Fig:CallgraphAfter}
\end{figure}

\begin{figure}[htbp]
\lstinputlisting[language={}, numbers=none]{code/callgraph-example/trace.txt}
\caption{The generated trace, using \texttt{code\_size()} as the entry point.
Some events are omitted for brevity.}
\label{Fig:CallgraphTrace}
\end{figure}

\begin{figure}[htbp]
\lstinputlisting[language={}, numbers=none]{code/callgraph-example/graph.txt}
\caption{The call graph (in JSON format) produced from the trace.}
\label{Fig:Callgraph}
\end{figure}

\section{Dealing with builtin and library functions}

The abovementioned instrumentation can't be applied to \matlab builtin
functions. It could potentially be applied to library functions, assuming their
source code was available and they were written in \matlab and not native code.
That being said, if the goal is to enable useful code navigation features, then
instrumenting library functions is of dubious utility. In any case, during the
course of a profiling run, control flow is likely to be passed to a builtin or
otherwise uninstrumented function, which could then call back into the project
code, for instance via a passed-in function handle, a method call on a
passed-in object, or the use of \matlab's reflective features, possibly using
passed-in arguments to compute names of functions or methods to call. In the
presence of such code, our approach is unsound; builtin functions calling back
into project code could manifest in the produced call graph as an edge linking
a call to a builtin function and a user-defined function, a handle to which was
passed in somewhere.

To preserve the soundness of our approach even in such cases, we rely on more
of \matlab's reflective features. In particular, the \texttt{functions} builtin
function allows us to inspect the contents of a function handle at runtime, and
determine which function it points to, the path to the file in which that
function was defined if applicable, and whether or not it's a builtin function.
Using this facility, we modify \texttt{mclab\_callgraph\_log\_then\_run} as
shown in \figref{Fig:LogThenRunBuiltin} to insert extra markers in the call
trace in order to distinguish calls to builtin functions. These markers are
ignored by the call graph construction step, but their presence in the trace
separates builtin call sites from user-defined function entrances, avoiding
the addition of the problematic edges.

\begin{figure}[htbp]
  \lstinputlisting[language=Matlab, numbers=none]{code/callgraph_builtins.m}
\caption{Code to handle builtins at runtime.}
\label{Fig:LogThenRunBuiltin}
\end{figure}


% TODO(isbadawi): See "Application-only Call Graph Construction", ECOOP 2012?

\section{Instrumentation performance overhead}

\section{Heuristics for minimizing call graph invalidation}

The call graph computed from a profiling run maps source positions to other
source positions. As such, it is extremely sensitive to even minor changes to
the project code -- even ostensibly inconsequential changes like adding
whitespace or comments would invalidate portions of the call graph as tokens
are shifted around. A sufficiently sophisticated implementation can ward
against this by keeping track of the moving tokens and computing offsets as
needed before querying the call graph. A larger issue is that, a priori, the
effect of a textual code change on the call graph is not obvious, requiring us
to update the call graph in the face of any change, no matter how minor, in
order to ensure stale edges are excised. Further, profiling doesn't really lend
itself to any sort of incremental call graph updating scheme; our only real
means of updating the call graph is to invalidate it and replace it with the
result of triggering another execution. Depending on the project and the
provided entry point code, each execution could potentially be expensive --
likely too expensive for our purposes, since in an IDE, unlike in some other
applications of call graphs, the code is apt to be in constant flux,
practically requiring us to be triggering an execution whenever the user stops
typing. Given all this, we would like to somehow make our call graph more
robust in the face of code changes.

\subsection{Multiple entry points}

Rather a relying on a single entry point to guide profiling runs, multiple
entry points can be provided, ideally each small and independent, analogously
to different tests in a unit test suite. During the profiling run, each entry
point is executed in turn, each resulting in its own call graph. These are then
combined into a composite call graph, queries against which are resolved by
forwarding them to each component call graph in turn and concatenating the
results together.

The advantage of this approach is that each entry point's call graph can be
invalidated independently. When a file is modified, only call graphs where
at least one node corresponds to a position within that file are discarded.
When queries are submitted to the composite call graph, it can attempt to
answer them with the remaining information. A profiling run only needs to be
triggered if the query comes up empty, and only the invalidated entry points
need to be executed again.

\subsection{Classifying edits}
% Try to determine just based on the edit whether recomputing is needed.
% If somehow we can determine that the callgraph is the same, then we just
% need to deal with tokens shifting around, which shouldn't be too bad.
% (Can maybe use Anchors in ace to track tokens moving).
%   e.g. Were function calls added or removed?
%   e.g. Are the arguments to the function calls the same?
%   Maybe do a diff of the AST?

% Invalidate only affected portions of the callgraph?

\section{Related work}

% Look into context-sensitive call graphs? Would be neat, but not sure what
% a good UI would be to exploit this information in the IDE...
% "Efficient Construction of Approximate Call Graphs for JavaScript IDE Services", ICSE 2013


\chapter{Static and dynamic code analysis} \label{chap:CodeAnalysis}
\input{text/code-analysis}

\chapter{Layout-preserving Refactorings} \label{chap:LayoutPreservingRefactorings}
A refactoring is a code transformation that changes the structure of the code
while preserving its semantics, and can often naturally be thought of as a
transformation over the structure of an abstract syntax tree. However, from
the perspective of a programmer using a refactoring tool, a refactoring is
ultimately a textual transformation. It is important to reconcile these two
conceptions of refactorings; purely working in terms of ASTs, while
technically correct from a semantics perspective, is apt to lose a lot
of information about the textual layout of the code, while purely working
in terms of text is apt to make implementations of individual refactorings
very brittle, hard to reason about, and hard to maintain.

In this chapter, we present our approach to the problem of implementing layout
preserving refactorings. We allow refactorings to be implemented in terms of a
minimalist tree transformation API, which hides the mechanics of layout
preservation. As refactoring writers specify tree-level changes, minimal
text-level changes are transparently computed from them. This simplifies the
implementations of individual refactorings, allowing them to remain oblivious
to the program text and to operate at a higher level of abstraction, without
compromising their practicality for end-users.

\section{Motivation}

Setting aside questions of semantics preservation, a refactoring is most
naturally thought of as a transformation on abstract syntax trees. For
instance, a refactoring like Extract Method can be boiled down to high-level
steps these:

\begin{enumerate}
  \item Synthesize a new function with the provided name
  \item Move the selected sequence of statements from the target function to
    the new function
  \item For each variable which is live at the input of the new function, add a
    corresponding input parameter to the function (or possibly or a global
    variable declaration)
  \item For each variable which is live at the end of the new function, add a
    corresponding output parameter to the function
  \item Synthesize a new function call with the appropriate number of input and
    output arguments
  \item Insert this function call in the target function, at the position of
    the original sequence of statements
\end{enumerate}

These steps are agnostic to program text. Given this, a natural structure for a
refactoring tool consists of parsing code, then performing transformations on
its AST, and then pretty printing the transformed AST to retrieve source code
to present to the programmer. Many refactoring tools -- particularly ones
developed for research purposes, where practicality is often a non-goal --
operate this way.

The problem with such approaches is that by construction the AST does not
contain enough information to accurately reconstruct the input source code.
Typically among the casualties are indentation and other whitespace, along
with comments and syntactic sugar.

\figref{Fig:LostLayout} shows a \matlab program and the result of parsing and
then pretty-printing it using the \mclab toolkit. The two programs are
behaviorally equivalent, but contain many syntactic differences. The comment
associated with the \code{cube} function has moved below the header. The
four-space indentation has been changed to two-space indentation. Each binary
expression has been wrapped in parentheses. The output parameters have been
wrapped in square brackets. The nested function \code{square} has been moved
-- in \matlab, nested functions have the same scope irrespective of where they
are declared, so during parsing they are all moved to the end of their enclosing
functions as a simplification step.

\begin{figure}
\begin{minipage}{0.5\linewidth}
\lstinputlisting[numbers=none]{code/cube_original.m}
\end{minipage}
\hfill \hspace{.3cm} \hfill
\begin{minipage}{0.5\linewidth}
\lstinputlisting[numbers=none]{code/cube_roundtrip.m}
\end{minipage}
\caption{An example of the lossy parsing and unparsing roundtrip.}
\label{Fig:LostLayout}
\end{figure}

Users of an automated refactoring tool are unlikely to be accepting of such
invasive changes to a program's text. As such, it is important for any
refactoring tool to be aware of the layout of the program when performing
refactorings. For a given AST transformation, it should endeavor to perform the
minimal textual changes needed to reflect the transformation in the program
text. In particular, unaffected portions of the program should not undergo any
textual changes.

Despite this, it is still convenient to express refactorings as tree
transformations. Refactorings would be much harder to implement and maintain if
they had to be expressed as textual transformations, or as a mixture of tree
and text transformations that had to be kept in sync.

Our goal is to be able to implement refactorings purely as tree
transformations, and to have minimal textual changes automatically computed
from them. In order to accomplish this, we introduce a simple transformation
API, which exposes a small set of tree manipulation operations. Instead of
directly manipulating AST nodes, refactorings are implemented in terms of this
API. Behind the scenes, the implementation of the API includes logic that keeps
track of the input program text in addition to the AST, and keeps the two in
sync.

\section{The transformation API}

Intuitively, all AST manipulation can be boiled down to a series of node
deletions and insertions. A replace operation is also convenient, and as we
will discuss further in the next section, our approach requires us to also take
on the responsibility of copying nodes. Finally, an operation to recover the
transformed source code is also needed. \figref{Fig:TransformerAPI} shows how
these operations are encoded as a Java interface. In order to demonstrate that
these operations are just conventional tree manipulation operations,
\figref{Fig:BasicTransformer} shows a trivial implementation of the interface
that just falls back on the operations exposed by the AST, ignoring layout
concerns; note the close correspondence between the two sets of operations.

\begin{figure}[htbp]
\lstinputlisting[numbers=none, language=Java]{code/Transformer.java}
\caption{The Transformer API, encoded as a Java interface}
\label{Fig:TransformerAPI}
\end{figure}

\begin{figure}
\lstinputlisting[numbers=none, language=Java]{code/BasicTransformer.java}
\caption{A trivial implementation of the Transformer API}
\label{Fig:BasicTransformer}
\end{figure}

\section{Synchronizing ASTs and token streams}

At a high level, our approach to layout preservation works as follows. Given a
\matlab source file, we start by tokenizing the source code using a \matlab
lexer, yielding a stream of tokens. Notably, this lexer does not drop any
tokens, preserving both whitespace and comments. Since some \matlab constructs
are whitespace-sensitive, and since keeping comments intact when compiling
\matlab to another language was considered a useful feature, the \mclab toolkit
already includes such a lexer; however, adapting this approach to other
languages may involve the use of a specialized lexer.

Alongside the token stream, we use a \matlab parser to parse the same source
file, yielding an abstract syntax tree. Our aim is to allow refactoring
writers to specify edits to the abstract syntax tree, and to have those edits
be transparently reflected in the token stream. In the end, when the time
comes to present the transformed source back to the user, we can simply
concatenate all the tokens instead of pretty printing the AST.

In order for this to work, there are two "primitives" that we rely on. First,
we need to be able to identify, for a particular AST node (which may be a node
from the original program, or a copy of a node, or a brand new synthesized
node), the portion of the token stream corresponding to that node. Second, we
need to be able to make local modifications to just that portion of the stream,
without compromising our ability to later look up nodes in the modified stream.

To bridge the gap between the token stream and the AST, we use position
information. Each token consists of a fragment of text together with a line and
column position where it occurs in the source code. Each AST node, assuming it
was produced by the parsing process and not manually synthesized after the
fact, also contains position information. As an initial link between the source
text and the AST, we can simply create a table that maps line and column
positions to the corresponding token in the stream. When we need to retrieve
the portion of the token stream for a given node, we can simply look up the
token corresponding to its start position, look up the node token corresponding
to its end position, and take all the tokens in between.

One potential complication here is that the mapping could become stale as the
token stream is modified. For instance, if we were to simply keep an array of
tokens and map positions to indices into the array, then as the stream is
edited and tokens are shifted around, indices would no longer point to the same
token. In some cases we may be able to update the mapping as we edit the
stream, but that approach quickly becomes brittle and hard to reason about.

In order to avoid this complication, and also to satisfy our requirement of
supporting local edits to the token stream, it is necessary to carefully
consider the data structure we will use to represent the token stream. A
natural choice is a doubly linked list. The values in our table can be
references to individual nodes in the list, which will remain valid even as
their positions within the list change. Also, since each node contains a
reference to its predecessor and successor within the list, we can cheaply
support edits like removing a sequence of tokens, or inserting a sequence of
tokens before or after a particular token.

A common operation when implementing refactorings is to copy an AST
node. Since this approach associates mutable state (a portion of the token
stream) with each node, it's not enough to simply copy the node; the
corresponding token stream fragment should be also be copied, and the copy
associated with the newly copied node, in order to ensure that we can clearly
distinguish the original from the copy and manipulate them independently.
This is the motivation for including the copy operation in the Transformer
interface shown in \figref{Fig:TransformerAPI}.

That suffices for correlating the token stream with the AST of the original
source text, but we need to be able to maintain this mapping as the token
stream is edited, code is moved around or copied, and new code is synthesized.

\section{Dealing with freshly synthesized code}

It is common for refactorings to insert new code into the program which wasn't
present in the original source text. For example, in the case of extract
method, a new function call is synthesized to replace the extracted statements,
a new function is synthesized to hold them, and that new function might also
contain some synthetic statements like global variable declarations to ensure
semantics are preserved. These pose a problem since there is no original text
to tokenize in this case.

The natural intuition is to somehow leverage the output of the pretty printer
to recover some text that we may integrate into the token stream. If we simply
pretty print the new AST, we get a program fragment as a string that we can
then feed to the lexer. However, since these synthetic AST nodes don't have
position information, we can't easily associate nodes with their portion of the
token stream. We can associate the top-level tree with the entire fragment, but
subtrees pose a problem.

One way to deal with this would be to pretty print the new AST to recover the
program text, then parse the text again to get back an AST that has position
information, and then proceed as before. This is viable, but it implies that
all new nodes would have to be synthesized through the tree transformation API
-- nodes that are synthesized by the caller directly couldn't be used, since
they wouldn't have the necessary position information.

In the interest of keeping the API surface small, we instead implement a
tokenizing pretty printer, which is a version of the pretty printer that emits
sequences of tokens instead of entire strings. This can implemented as a
straightforward recursive traversal of the AST; for each node, we synthesize a
sequence of tokens, and at the same time associate that sequence with the node
for later use.

\section{Putting it all together}

Given a program $P$, the lexer produces a stream $S$ of tokens $t_1, \dots,
t_n$. For a given token $t_i$ in $S$, we write $text(t_i)$ for the token's text
content, \emph{startpos}$(t_i)$ for the token's starting position, and
\emph{endpos}$(t_i)$ for the token's ending position. Given the same $P$, the
parser produces an AST $T$. For each AST node $n$ in $T$, we overload our
previous definitions and write \emph{startpos}$(n)$ for the node's starting position,
and \emph{endpos}$(n)$ for the node's ending position, which are the same as the
starting and ending positions of the first and last tokens corresponding to $n$,
respectively. We take all of these as inputs; these are conventionally provided
by a typical lexer and parser.

We start by creating a doubly linked list $L$ with a node for each token in the
token stream. We write $head(L)$ and $tail(L)$ for the head and tail nodes of
$L$, respectively. Given a node $n$, we write $token(n)$ for the token
associated with $n$, $prev(n)$ for its predecessor node in $L$, and $next(n)$
for its successor node in $L$. We then define a mapping $P$ from source
positions to nodes in $L$; for each node $n$ in $L$, we map both
\emph{startpos}$(token(n))$ and \emph{endpos}$(token(n))$ to $n$.

A token stream fragment $f = \langle start(f), end(f) \rangle$ is a pair of
nodes in $L$: a starting node $start(f)$ and an ending node $end(f)$. Given an
AST node $n$ in $T$, we ultimately need to be able to retrieve (or synthesize)
a corresponding token stream fragment. For a node $n$ occurring in the original
AST produced by the parser, the corresponding token stream fragment is $\langle
P[$\emph{startpos}$(n)], P[$\emph{endpos}$(n)] \rangle$, but as alluded to in
the previous sections, this won't be accurate when dealing with synthetic
nodes, or copies of nodes. To handle these, we allow a node's token stream
fragment to be set explicitly, skipping the lookup in $P$. This facility is
used by the tokenizing pretty printer, and in the implementation of the copy
operation. We write $fragment(n)$ to retrieve the token stream fragment
associated with node $n$, if any.

Now when it comes to getting at a token stream fragment corresponding to a
given AST node $n$, we distinguish between the following cases.

\begin{enumerate}
  \item $n$ is a node from the original program, existing in the original
    AST produced by the parser. In that case, the corresponding fragment
    is just $\langle P[$\emph{startpos}$(n)], P[$\emph{endpos}$(n)] \rangle$.
  \item $n$ is a synthetic node we're seeing for the first time. In that case,
    we invoke the tokenizing pretty printer on $n$, which will output another
    doubly linked list $L'$. Now the corresponding fragment is
    $\langle head(L'), tail(L') \rangle$.
  \item $n$ is a synthetic node we've seen before. In that case, the corresponding
    fragment is $fragment(n)$.
  \item $n$ is a copy of another node. In this case, the corresponding fragment is
    again $fragment(n)$.
\end{enumerate}

\section{Heuristics for handling indentation and comments}

Our approach of correlating AST nodes with fragments of the original program
text and performing local edits to the token stream lets us easily preserve the
layout of the unaffected portions of the code, as well as the internal layout
of the affected portions. However, care must still be taken when dealing with
code at the boundary between affected and unaffected. For instance, when
inserting a statement in the body of a deeply nested control structure, we
should pick an appropriate amount of indentation to match the surrounding code
-- but the program indentation might be inconsistent, or there might not be any
surrounding code in a particular file. Statements can also have comments
associated with them, and ideally these should be moved alongside their
subjects -- but since comments are largely free-form, there is no easy way to
identify which comments are associated with a given statement. In both cases,
we rely on heuristics to try and do something sensible.

\subsection{Indentation}

When moving statements or functions, we attempt to match the surrounding
indentation. Given a simple statement node $n$, we compute its indentation
level by taking $start(fragment(n))$ and searching backwards along its
predecessor nodes until a non-whitespace or newline token $t$ is reached. Then
the token stream fragment corresponding to the indentation is \newline $\langle
next(t), prev(start(fragment(n))) \rangle$ (which is possibly empty). For a
multiline construct like a function or a loop, we compute the indentation of
each line and take the common prefix. When inserting a node, we try to guess
the appropriate amount of indentation to insert by applying the following
heuristic.

\begin{enumerate}
  \item Special case: if we're inserting a statement between two semicolon or
    comma separated statements on the same line, then there's no indentation to
    add.
  \item Otherwise, look for a predecessor or successor AST node and copy its
    indentation.
  \item Otherwise, look for a statement or function list elsewhere in the file,
    and copy the indentation of the first node there.
  \item Otherwise, there is nothing to go on -- a more sophisticated approach
    might inspect different files in the project, but each token stream is tied
    to a single file -- so we fall back to an arbitrary (potentially
    configurable) default of four spaces.
\end{enumerate}

\subsection{Comments}

In order to represent an association between a comment and a language
construct, we simply extend the token stream fragment associated with the
construct to include the comment. Note that this implies the portion of the
program text including the construct and the comment must be contiguous,
which might be problematic; for example, consider the (artificial) case
in \figref{Fig:WackyComments}, where multiple aligned inline comments are
meant to be associated with the first statement. We choose not to handle
such cases; at the least, it would disproportionately complicate the
implementation, as each node might have several disparate token stream
fragments associated with it.

\begin{figure}
\begin{lstlisting}[numbers=none, keepspaces=true]
function_call()  % This comment spans multiple lines
another_call()   % and lies adjacent to many statements
some_core_code() % but should be considered associated with
                 % the first function call.
\end{lstlisting}
\caption{An illustration of wacky commenting practices.}
\label{Fig:WackyComments}
\end{figure}

Comments can be associated with any kind of language construct, including
functions, scripts, classes, methods, statements and even individual
expressions. Comments associated with a node are also implicitly associated
with its ancestor nodes, so that, for instance, moving a function call also
moves any comments associated with its arguments. In each case, when an
operation is performed on a node, we inspect the surrounding code to gather any
associated comments. Given a node, our heuristic is as follows.

\begin{enumerate}
  \item For a statement or an expression, we include the trailing inline
    comment, if any. Given the node $n$, we take $end(fragment(n))$, which is a
    node in $L$, and search forwards along its successor nodes, skipping over
    any space or tab (but not newline) tokens. If we reach a comment, which the
    lexer will have grouped into a single token $t$, we replace
    $end(fragment(n))$ by $t$. Otherwise we'll reach either a newline, the end
    of the file, or some other text token, in which case we stop.
  \item For all nodes, we also include any preceding line comments. Given
    a node $n$, we take $start(fragment(n))$, which is a node in $L$, and
    search backwards along its predecessor nodes, skipping any whitespace
    tokens, including newline tokens. For each line comment token $t$ we
    encounter, we replace $start(fragment(n))$ by $t$ and continue, stopping
    only when we encounter a text token, or the start of the file.
\end{enumerate}

\section{Niggling details: delimiters, parentheses}

In addition to moving existing pieces of code around and dealing with newly
created code, in some cases the transformation engine also needs to synthesize
new code.

Various \matlab language constructs are represented in the AST by delimited
lists. For instance, function bodies are newline or semicolon delimited lists
of statements, function definitions contain comma-delimited lists of input and
output parameter names, and array accesses or function calls contain comma or
space delimited lists of arguments. Since these delimiters are purely
syntactic, they are not represented in the AST. As such, in order to support
manipulating delimited lists, the transformation engine needs to transparently
deal with delimiters. For example, adding a second argument to a function call
-- represented in code by adding an expression node to a list of expressions --
requires inserting a comma before the text corresponding to the expression.
Similarly, removing an argument requires removing the corresponding comma.

Since the Transformer API takes all parameters as \texttt{ASTNode<?>}, the root
of the AST node class hierarchy, it doesn't necessarily know which delimiters
to use. One way to address this would be to have the Transformer API expose
different operations to manipulate statement lists, argument lists, and
parameter lists, but that would entail a much larger API surface -- a
disportionate cost for what should be a minor concern. Instead, the
transformation engine handles this by inspecting the AST fragments handed to
the \texttt{insert} and \texttt{remove} operations. By inspecting the structure
of the AST and performing some runtime checks to determine the context we're
operating in, we can distinguish between the different cases we need to handle.
For example, \figref{Fig:WhereAreWe} shows a small method we can use to tell
whether an arbitrary AST node actually represents the input parameter list of a
function; in that case, we would know to use commas as delimiters, and also to
surround the list with square brackets as needed.

\begin{figure}
\begin{lstlisting}[language=Java, numbers=none]
private boolean isInputParamList(ASTNode<?> node) {
  return node.getParent() instanceof Function &&
      ((Function) node.getParent()).getInputParams() == node;
}
\end{lstlisting}
\caption{Using runtime checks and AST traversal to examine the context.}
\label{Fig:WhereAreWe}
\end{figure}

Another case where the transformation engine may need to synthesize extra code
is the preservation of operator precedence. \figref{Fig:NeedParens} shows
a motivating example -- if we want to inline the \texttt{x} variable, the
expression on the right hand side needs to be wrapped in parentheses. This
should ideally be transparent to the calling code, where this should be simply
a call to the \texttt{replace} operation. As a simple heuristic to handle cases
like these, the transformation engine checks during the \texttt{copy} operation
whether the copied node is a binary expression, and wraps it with parentheses
if they're not already there.

\begin{figure}
\begin{minipage}{0.5\linewidth}
\lstinputlisting[numbers=none]{code/inline_parens_original.m}
\end{minipage}
\hfill \hspace{.3cm} \hfill
\begin{minipage}{0.5\linewidth}
\lstinputlisting[numbers=none]{code/inline_parens_transformed.m}
\end{minipage}
\caption{An example illustrating the need for synthesized parentheses.}
\label{Fig:NeedParens}
\end{figure}

\section{Case studies: inline variable, extract function}

The Inline Variable refactoring is relatively simple; it takes an assignment
statement as input, and replaces each use of the assigned variable with the
expression on the right hand side before removing the assignment.
\figref{Fig:InlineVariableComparison} shows a pre-existing implementation of this
refactoring (skipping over the portions of the code dealing with the
correctness of the transformation), followed by an equivalent implementation against
the transformation API.

% TODO(isbadawi): Add before-after figure, showing ASTs and token streams.

\begin{figure}
\begin{minipage}{\linewidth}
\lstinputlisting[numbers=none, language=Java]{code/InlineVariableBefore.java}
\end{minipage}
\begin{minipage}{\linewidth}
\lstinputlisting[numbers=none, language=Java]{code/InlineVariable.java}
\end{minipage}
\caption{The original implementation of the Inline Variable refactoring using plain
AST operations, and the new implementation against the Transformer API.}
\label{Fig:InlineVariableComparison}
\end{figure}

The Extract Function refactoring slightly more involved.
\figref{Fig:ExtractFunction} shows how the mechanics of the transformations are
implemented against the transformation API. Even though the refactoring moves
AST nodes around, copies nodes, and mixes in synthetic code with the original
text, the code is completely oblivious to text, instead leaning on the {\tt
Transformer} to do the heavy lifting.

\begin{figure}
\lstinputlisting[numbers=none, language=Java]{code/ExtractFunction.java}
\caption{The implementation of the Extract Function refactoring using the transformer API}
\label{Fig:ExtractFunction}
\end{figure}

\section{Related work}

The work that most closely resembles ours is HaRe, a refactoring tool for
Haskell \cite{HaRe}. It uses a similar approach of synchronizing an AST with a
token stream; program analyses are carried out using the AST, but program
transformations are carried out on the token stream alongside the AST, and
source positions are used to bridge the two. Rather than using a tokenizing
pretty printer for synthesized code, the output of the regular pretty printer
is used, together with a scheme where the concatenated tokens are re-lexed to
obtain correct positions which are then used to update the position information
of existing AST nodes. Similar heuristics for associating comments with program
structures are also presented.

De Jonge and Visser \cite{AlgorithmForLayoutPreservation} present an algorithm
for layout preservation in refactoring transformations. While the token stream
is used to access the layout structure surrounding a given AST node, it is not
modified in parallel with the AST; instead, the source code is reconstructed
afterwards using a combination of the original program text, pretty printed
text, and the application of a tree differencing algorithm to detect insertions
and deletions. An attempt is made to formalize the problem and prove the
algorithm correct within that formalization; by contrast, we are largely
describing an implemention.

Waddington and Yao \cite{Proteus} tackled the same problem, which they termed
"the problem of style disruption", with Proteus, their refactoring tool for C
and C++. Their approach was to use a specialized AST called a "Literal-Layout
AST (LL-AST)", where literals, token and whitespace nodes are interspersed
alongside the regular nodes. Enhancing the AST with layout information has also
been the theme of a few other approaches to this problem \cite{RefactorErl}.

% TODO(isbadawi): Citation?
The Eclipse JDT contains infrastructure for modifying code at two levels -- a
lower-level API for describing text manipulation primitives, and a higher-level
AST rewriting API, which accepts descriptions of changes to AST nodes and uses
the text manipulation API to try and perform the textual changes required to
represent the AST changes. The approach is similar to ours in spirit; one big
difference is that since that our approach is implemented largely as a
standalone tool, we rely solely on lexing and parsing as primitives, while
Eclipse's implementation benefits from more sophisticated integration with a
scriptable text editor.


\chapter{Related Work} \label{chap:RelatedWork}
\input{text/related-work}

\chapter{Conclusions and Future Work} \label{chap:Conclusions}
This thesis introduced McIDE, a \matlab IDE powered by the \mclab compiler
toolkit, and with a focus on powering features through exploiting runtime
information rather than relying solely on static analysis. We provided an
overview of McIDE's design, which consists of largely independent components
wired together through a thin browser-based graphical interface. We described
our dynamic call graph collection mechanism, and the analyses and optimizations
we implemented to minimize the performance overhead of the instrumented code.
We presented a technique for performing code transformations in a
layout-preserving fashion, which allows McIDE to provide some usable automated
refactorings out of the box, and future refactoring implementers to reuse the
transformation infrastructure to do the same. Finally, we described \matlab's
dynamic features in detail and presented a study of their usage in the wild,
later exploiting some of our findings to automatically eliminate their uses
where possible.

\section{Future Work}

We present here some ideas for possible further work for the continued
development of McIDE. The common thread is finding more useful ways to exploit
runtime information.

\begin{description}

\item[Dynamic code visualizations] IDEs are in a position to provide alternate
perspectives on the code beyond the traditional file explorer view, and runtime
information could be particularly useful in this setting.

As low hanging fruit, the call graph information described in this thesis could
be used to show execution stack traces as a tree and enable jumping directly to
any function invoked along the way.

Relevant runtime information could be overlaid onto source code in order to aid
program understanding -- expressions could be annotated with runtime types (or
even values), perhaps via tooltips or comments introduced into the code if
requested; lines of code could be annotated with timing information, for
instance by coloring bottlenecks differently as to make them stand out; the
profiling machinery could work backwards from errors in the execution to
highlight the problematic code paths.

\item[More sophisticated dynamic or blended analyses] We restricted ourselves
to uses of dynamic features in this thesis, but there is apt to be many more
ways in which \matlab code in the wild can be automatically improved.

\item[Better integration with \mclab static backends] The dynamic feature
elimination techniques presented in \chapref{chap:DynamicFeatureElimination}
make the code more amenable to static analysis, and thus more suitable as input
to static backends of the \mclab toolkit, which operate on the \matlab subset
supported by the \matlab Tamer, which rules out features like \code{eval}. If
static compilation is a goal for the user, then more work is this vein could
help compatibility along.

\end{description}


\appendix % -- Appendices -----------------------------------------------------

\addtocontents{toc}{\protect\addvspace{10pt}}
\addtocontents{toc}{\protect\contentsline{part}{Appendices}{}{}}

% -- Bibliography -------------------------------------------------------------

%\addtocontents{toc}{\protect\addvspace{10pt}}

%\bibliographystyle{plain}

\bibliographystyle{web-alpha} %- originally in jesse thesis
%\bibliography{strings, thesis}
\bibliography{citations}

% -- Glossary & Index ---------------------------------------------------------

%\addtocontents{toc}{\protect\addvspace{10pt}}
%\include{text/appendices/Glossary}
%\include{text/appendices/Index}

\end{document}
